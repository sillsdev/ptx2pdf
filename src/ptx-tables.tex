%%%%%%%%%%%%%%%%%%%%%%%%%%%%%%%%%%%%%%%%%%%%%%%%%%%%%%%%%%%%%%%%%%%%%%%
% Part of the ptx2pdf macro package for formatting USFM text
% copyright (c) 2007-2008 by SIL International
% written by Jonathan Kew
%
% Permission is hereby granted, free of charge, to any person obtaining  
% a copy of this software and associated documentation files (the  
% "Software"), to deal in the Software without restriction, including  
% without limitation the rights to use, copy, modify, merge, publish,  
% distribute, sublicense, and/or sell copies of the Software, and to  
% permit persons to whom the Software is furnished to do so, subject to  
% the following conditions:
%
% The above copyright notice and this permission notice shall be  
% included in all copies or substantial portions of the Software.
%
% THE SOFTWARE IS PROVIDED "AS IS", WITHOUT WARRANTY OF ANY KIND,  
% EXPRESS OR IMPLIED, INCLUDING BUT NOT LIMITED TO THE WARRANTIES OF  
% MERCHANTABILITY, FITNESS FOR A PARTICULAR PURPOSE AND  
% NONINFRINGEMENT. IN NO EVENT SHALL SIL INTERNATIONAL BE LIABLE FOR  
% ANY CLAIM, DAMAGES OR OTHER LIABILITY, WHETHER IN AN ACTION OF  
% CONTRACT, TORT OR OTHERWISE, ARISING FROM, OUT OF OR IN CONNECTION  
% WITH THE SOFTWARE OR THE USE OR OTHER DEALINGS IN THE SOFTWARE.
%
% Except as contained in this notice, the name of SIL International  
% shall not be used in advertising or otherwise to promote the sale,  
% use or other dealings in this Software without prior written  
% authorization from SIL International.
%%%%%%%%%%%%%%%%%%%%%%%%%%%%%%%%%%%%%%%%%%%%%%%%%%%%%%%%%%%%%%%%%%%%%%%

% ptx-tables.tex
% support for USFM table markup

\addtoinithooks{\set@ptablemarkers}

\def\@settablemark#1#2{\x@\def\csname #1\endcsname{\t@blecell{#1}{#2}}\structm@rker{#1}}

\m@kedigitsletters
\def\set@ptablemarkers{% redefine the markers from usfm.sty
  \@settablemark{th1}{l}
  \@settablemark{th2}{l}
  \@settablemark{th3}{l}
  \@settablemark{th4}{l}
  \@settablemark{th5}{l}
  \@settablemark{th6}{l}
  \@settablemark{th7}{l}
  \@settablemark{th8}{l}
  \@settablemark{th9}{l}
  \@settablemark{thc1}{c}
  \@settablemark{thc2}{c}
  \@settablemark{thc3}{c}
  \@settablemark{thc4}{c}
  \@settablemark{thc5}{c}
  \@settablemark{thc6}{c}
  \@settablemark{thc7}{c}
  \@settablemark{thc8}{c}
  \@settablemark{thc9}{c}
  \@settablemark{thr1}{r}
  \@settablemark{thr2}{r}
  \@settablemark{thr3}{r}
  \@settablemark{thr4}{r}
  \@settablemark{thr5}{r}
  \@settablemark{thr6}{r}
  \@settablemark{thr7}{r}
  \@settablemark{thr8}{r}
  \@settablemark{thr9}{r}
  \@settablemark{tc1}{l}
  \@settablemark{tc2}{l}
  \@settablemark{tc3}{l}
  \@settablemark{tc4}{l}
  \@settablemark{tc5}{l}
  \@settablemark{tc6}{l}
  \@settablemark{tc7}{l}
  \@settablemark{tc8}{l}
  \@settablemark{tc9}{l}
  \@settablemark{tcc1}{c}
  \@settablemark{tcc2}{c}
  \@settablemark{tcc3}{c}
  \@settablemark{tcc4}{c}
  \@settablemark{tcc5}{c}
  \@settablemark{tcc6}{c}
  \@settablemark{tcc7}{c}
  \@settablemark{tcc8}{c}
  \@settablemark{tcc9}{c}
  \@settablemark{tcr1}{r}
  \@settablemark{tcr2}{r}
  \@settablemark{tcr3}{r}
  \@settablemark{tcr4}{r}
  \@settablemark{tcr5}{r}
  \@settablemark{tcr6}{r}
  \@settablemark{tcr7}{r}
  \@settablemark{tcr8}{r}
  \@settablemark{tcr9}{r}
  \let\@TR=\tr
  \let\tr=\t@blerow
}
\m@kedigitsother

\def\t@blerow{%
  \startt@ble
  \@ndcell
  \@ndrow
  \global\advance\tabler@ws by 1
  \tablec@l=0
  \@tablerowtrue      
   %\@TR 
   %\p@rstyle{tr}%
  
   \s@tbaseline{\m@rker}{\styst@k}%
   \global\startparatrue
   \resetp@rstyle
   \p@rstyle@everypar{tr}%
   %\mcpop
   %\s@tstyst@k
  \@tablerowfalse
  \@rowtrue\trace{t}{rows=\the\tabler@ws}%
  \ignorespaces}

\addtoparstylehooks{\if@tablerow\else\endt@ble\fi}
\newif\if@tablerow
\newif\if@row
\newif\ifsp@ncols
\newbox\t@bleinserts

\def\@ndrow{%
  \if@row
    \loop \ifnum\tablec@l<\maxtablec@l
      \advance\tablec@l by 1
      \x@\let\x@\c@rrentcolbox\csname tablecol-\the\tablec@l\endcsname
      \setbox\c@rrentcolbox=\vbox{\parfillskip=0pt\hbox{}\unvbox\c@rrentcolbox}%
      \repeat
    \@rowfalse
    % store inserts in reverse row order for easy extraction later
    % most inserts are in 0ht vboxes so use 1sp is a flag for empty
    \setbox\t@bleinserts=\vbox{\parfillskip=0pt\ifvoid\t@blenotes\vbox to 1sp{}\else\box\t@blenotes\fi\unvbox\t@bleinserts}%
  \fi}

\catcode`\~=12 \lccode`\~=32
\lowercase{%
  \def\t@blecell#1#2{%
    \@ndcell
    \st@rtcell
    \xdef\mytmp{#2}%
    \getp@ram{justification}{#1\ifdiglot \c@rrdstat\fi}{#1+tr}%
    \trace{t}{justification of #1\ifdiglot \c@rrdstat\fi = \p@ram fallback \mytmp}%
    \s@ttc@lign
    \trace{t}{Define cell [\the\tabler@ws, \the\tablec@l] = \mytmp\space in style #1}%
    \ifnum\tablec@l>\maxtablec@l \global\maxtablec@l=\tablec@l \fi
    \ch@rstyle{#1}~\ignorespaces}%

  \def\sp@ncell#1#2#3#4{%#1-format, #2=justification, #3=start cell #4=end cell
    \@ndcell
    \let\@ndcell\@@ndspan
    \sp@ncolstrue
    \st@rtcell%\st@rtsp@n{#3}{#4}%
    \xdef\mytmp{#2}%default justification
    \trace{t}{justification of #1\ifdiglot \c@rrdstat\fi = \p@ram fallback \mytmp}%
    \getp@ram{justification}{#1}{#1+#1+tr}%
    \s@ttc@lign
    \trace{t}{Define spanning cell [\the\tabler@ws, \the\tablec@l - #4] = \mytmp\space in style #1}%
    \m@kenumber{#4}\x@\xdef\csname span-\the\tabler@ws-\the\tablec@l\endcsname{\@@result}%
    \ifnum\@@result>\maxtablec@l \global\maxtablec@l=\tablec@l \fi
    \ch@rstyle{#1}~\ignorespaces}%
}
\def\s@ttc@lign{%
    \ifx\p@ram\c@nter \edef\mytmp{c}\else
    \ifx\p@ram\l@ftb@l \edef\mytmp{l}\else
    \ifx\p@ram\l@ft \edef\mytmp{l}\else
    \ifx\p@ram\r@ght \edef\mytmp{r}%
    \fi\fi\fi\fi
    \x@\xdef\csname align-\the\tabler@ws-\the\tablec@l\endcsname{\mytmp}%
}

\newtoks\s@vedm@rks
\let\origm@rk\mark
\let\origm@rks\marks
\newcount\s@vedmarkcount
\s@vedm@rks{}
\def\savingmarks{%
    \global\advance\s@vedmarkcount by 1
    \def\mark##1{\x@\global\x@\s@vedm@rks\x@{\the\s@vedm@rks \origm@rk{##1}}}%
    \def\marks##1##2{\trace{t}{marks ##1, ##2}\x@\global\x@\s@vedm@rks\x@{\the\s@vedm@rks \origm@rks##1{##2}}}%
    \aftergroup\dos@vedm@rk%
}
\newdimen\t@blehsize \t@blehsize=0pt
\def\st@rtcell{%
  \advance\tablec@l by 1
  \ifnum\tablec@l>\allocatedtablec@ls \@llocnewc@l \fi
  \setbox\cellb@x=\vbox\bgroup
    \savingmarks
    \t@blehsize=\hsize \hsize=\maxdimen
    \resetp@rstyle \everypar={}\noindent
    %\ifRTL\beginR\fi
    \dimen255=0pt
    \edef\tc@ll{tc\the\tablec@l}%
    \getp@ram{leftmargin}{\tc@ll}{\tc@ll}%
    \ifx\p@ram\relax \else \trace{t}{leftmargin of \tc@ll is \p@ram *\the\IndentUnit}\advance\dimen255 \p@ram\IndentUnit \fi
    \getp@ram{firstindent}{\tc@ll}{\tc@ll}%
    \ifx\p@ram\relax \else \advance\dimen255 \p@ram\IndentUnit \fi
    \trace{t}{Kern for start of \the\tablec@l, \the\dimen255, firstindent=\p@ram}% 
    \ifdim\dimen255=0pt\else\kern\dimen255\fi
    \s@tfont{tc\the\tablec@l}{\tc@ll+\styst@k}%
    \global\advance\p@ranesting by 1
    \@celltrue}
\def\dos@vedm@rk{%
  \trace{ma}{Recovering \the\s@vedmarkcount marks,  gt:\the\currentgrouptype, gd:\the\currentgrouplevel}%
  \s@vedmarkcount=0
  \the\s@vedm@rks\relax\s@vedm@rks{}%
}

\def\@@@ndcell{%
    \unskip
    \getp@ram{rightmargin}{tc\the\tablec@l}{tc\the\tablec@l}%
    \global\let\p@r@m\p@ram %preserve against endgroups
    \end@llpoppedstyles{tc*}%
    \let\p@ram\p@r@m
    \ifx\p@ram\relax\else\ifx\p@ram\z@ero\else
      \trace{t}{rightkern(tc\the\tablec@l) \p@ram, from tc\the\tablec@l}\kern\p@ram\IndentUnit
    \fi\fi
    \parfillskip=0pt
    \egroup
    \setbox0=\vbox{\parfillskip=0pt\unvbox\cellb@x \global\setbox\cellb@x=\lastbox}%
    \global\setbox\cellb@x=\hbox{\unhbox\cellb@x \unskip\unskip\unpenalty}%
    \x@\let\x@\c@rrentcolbox\csname tablecol-\the\tablec@l\endcsname
}
\def\@@@ddcell{
    \ifvoid\c@rrentcolbox % adding a column, may need to add empty cells above
      \count255=1
      \loop \ifnum\count255<\tabler@ws \advance\count255 by 1
        \setbox\c@rrentcolbox=\vbox{\parfillskip=0pt\hbox{}\unvbox\c@rrentcolbox}%
        \repeat
    \fi
    \trace{t}{Add cell width \the\wd\cellb@x, to row \the\tabler@ws, col \the\tablec@l, width=\the\wd\c@rrentcolbox}%
    \setbox\c@rrentcolbox=\vbox{\parfillskip=0pt\box\cellb@x\unvbox\c@rrentcolbox}%
      % note that this stacks the cells upwards! they'll be reversed later
}
\def\@@ndcell{%
  \if@cell
    \@@@ndcell
    \@@@ddcell
  \fi}

\def\@@ndspan{%
  \if@cell
    \@@@ndcell
    \wd\cellb@x=0pt
    \@@@ddcell
  \fi
  \x@\let\x@\sp@nend\csname span-\the\tabler@ws-\the\tablec@l\endcsname
  \loop
    \advance\tablec@l by 1
    \ifnum\tablec@l>\allocatedtablec@ls \@llocnewc@l \fi
    \ifnum\tablec@l<\sp@nend\repeat
  \global\let\@ndcell\@@ndcell
}

\let\@ndcell\@@ndcell
\def\spanningcell#1#2#3#4{\x@\def\csname #1#2#3-#4\endcsname{\sp@ncell{#1#2#3-#4}{#2}{#3}{#4}}}
\let\spanningcols\spanningcell

\newif\if@cell
\newbox\cellb@x
\def\tablecategory{}

%\tracinggroups=1
%\tracingassigns=1
\def\startt@ble{%
  \ifdoingt@ble \else
    \ifhe@dings\endhe@dings\fi
    \endlastp@rstyle{tr}%
    \trace{t}{starting table from \m@rker}%
    \op@ninghooks{between-\m@rker}{tr}{tr\ifx\styst@k\empty\else+\styst@k\fi}% Run any between old par an current
    \let\stylet@pe\ss@Tbl
    \mcpush{\stylet@pe}{tr}%
    \s@tstyst@k
    \bgroup
    \sp@ncolsfalse % (so far) no column-spanning cells
    \tabler@ws=0
    \tablec@l=0
    \global\maxtablec@l=0
    \doingt@bletrue
    \message{starting table cat:\tablecategory \c@tegory}%
    \ifx\tablecategory\empty\else\global\doc@t{\tablecategory}\fi
  \fi}
\newdimen\t@talwidth
\def\endt@ble{%
  \ifdoingt@ble
    \trace{t}{endt@ble-\mcstack}%
    \@ndcell\@ndrow
    \t@talwidth=0pt
    \f@reachcol{\edef\tc@l{tc\the\@col}\getp@ram{\ifRTL leftmargin\else rightmargin\fi}{\tc@l}{\tc@l+\styst@k}\ifx\p@ram\relax \else
        \trace{t}{oldcolwidth \the\@col \space is \the\wd\c@rrentcolbox + \p@ram * \the\IndentUnit}%
        \dimen0=\wd\c@rrentcolbox\advance\dimen0 \p@ram\IndentUnit\wd\c@rrentcolbox=\dimen0
      \fi\trace{t}{colwidth \the\@col \space is \the\wd\c@rrentcolbox}\advance\t@talwidth by \wd\c@rrentcolbox}%
    \endt@blewrap
    \f@reachcol{\setbox\c@rrentcolbox=\box\voidb@x}%
    \end@llpoppedstyles{\ss@Tbl*}%
    \doingt@blefalse
    \egroup
    \let\tablecategory\empty
  \fi}

\let\zendtable\endt@ble

\newdimen\trmargin \trmargin=0pt
\def\endt@blewrap{%
  \trace{t}{endt@blewrap cols=\the\maxtablec@l, rows=\the\tabler@ws, hsize=\the\hsize, tw=\the\t@talwidth}%
  % calculate column widths as:
  % (a) any column whose max is < \hsize / numcols gets its natural width
  % (b) calculate exc@ss width of the remaining columns
  % (c) calculate sp@re width available from the narrow ones
  % (d) distribute this in proportion to the relative exc@ss of each
  \sp@re=0pt \exc@ss=0pt
  \thr@shold=\hsize
  \getp@ram{leftmargin}{tr}{\styst@k}\ifx\p@ram\relax\else\advance\thr@shold by -\p@ram\IndentUnit\fi 
  \getp@ram{rightmargin}{tr}{\styst@k}\ifx\p@ram\relax\else\advance\thr@shold by -\p@ram\IndentUnit\fi 
  \divide\thr@shold by \maxtablec@l
  \trace{t}{thr@shold=\the\thr@shold, hsize=\the\hsize, textwidth=\the\textwidth, max:\the\maxtablec@l}%
  \f@reachcol{%
    \edef\tc@l{tc\the\@col}%
    \getp@ram{spacebefore}{\tc@l}{\tc@l+\styst@k}% abuse spacebefore as fixed column width
    \ifx\p@ram\relax\c@rrentcolwidth=-1sp \else\ifdim\p@ram\textwidth=0pt \c@rrentcolwidth=-1sp \else\c@rrentcolwidth=\p@ram\textwidth\divide\c@rrentcolwidth 12\trace{t}{width=\the\c@rrentcolwidth}\fi\fi
    \ifdim\wd\c@rrentcolbox>\thr@shold
      \trace{t}{Adding \the\wd\c@rrentcolbox \space to excess (\the\exc@ss)}%
      \advance\exc@ss by \wd\c@rrentcolbox \advance\exc@ss by -\thr@shold
    \else
      \ifdim \c@rrentcolwidth=-1sp\relax
        \c@rrentcolwidth=\wd\c@rrentcolbox
        \edef\tc@ll{tc\the\@col}%
        \getp@ram{leftmargin}{\tc@ll}{\tc@ll}%
        \ifx\p@ram\relax \else \advance \c@rrentcolwidth\p@ram \IndentUnit\fi
        \getp@ram{rightmargin}{\tc@ll}{\tc@ll}%
        \ifx\p@ram\relax \else \advance \c@rrentcolwidth\p@ram \IndentUnit\fi
      \else
        \bgroup\count255=\c@rrentcolwidth \trace{t}{c@rrentcolwidth is \the\count255 sp}\egroup%
      \fi
      \trace{t}{Adding \the\thr@shold - \the\c@rrentcolwidth (\the\wd\c@rrentcolbox) \space to spare (\the\sp@re)}%
      \advance\sp@re by \thr@shold \advance\sp@re by -\c@rrentcolwidth
    \fi
  }%
  \trace{t}{thr@shold=\the\thr@shold, exc@ss=\the\exc@ss, sp@re=\the\sp@re}%
  \f@reachcol{%
    \ifdim\c@rrentcolwidth=-1sp
      \ifdim\exc@ss>\sp@re % we have to wrap at least one column
        \c@rrentcolwidth=\wd\c@rrentcolbox \advance\c@rrentcolwidth by -\thr@shold
        \ifdim\c@rrentcolwidth>1638pt \tempfalse\message{Table column overflow, reducing resolution}\else\temptrue\fi
        \iftemp\multiply\c@rrentcolwidth by 20\fi
        \divide\c@rrentcolwidth by \exc@ss
        \multiply\c@rrentcolwidth by \sp@re
        \iftemp\divide\c@rrentcolwidth by 20\fi
        \advance\c@rrentcolwidth by \thr@shold
      \else % there was enough sp@re, just give columns their desired width
        \c@rrentcolwidth=\wd\c@rrentcolbox
      \fi
    \fi
    \trace{t}{Table col width for \the\@col: \the\c@rrentcolwidth}%
  }%
%  \f@reachcol{%
%    \MSG{column \the\@col: width=\the\c@rrentcolwidth\space
%      \ifdim\c@rrentcolwidth<\wd\c@rrentcolbox
%        (wrapped, natural was \the\wd\c@rrentcolbox)\fi}%
%  }%
  \getp@ram{justification}{tr}{\styst@k}%
  \ifx\p@ram\r@ght
    \ifRTL\RTLfalse\else\RTLtrue\fi
  \fi
  \f@reachcol{\wr@pcolumn}%Wrap the cells in each columns.
  \dimen0=0pt \trmargin=0pt
  \getp@ram{leftmargin}{tr}{\styst@k}%
  \ifx\p@ram\relax\else\advance\ifRTL\trmargin\else\dimen0\fi by \p@ram\IndentUnit\fi
  \getp@ram{rightmargin}{tr}{\styst@k}%
  \ifx\p@ram\relax\else\advance\ifRTL\dimen0\else\trmargin\fi by \p@ram\IndentUnit\fi
  \trace{t}{Row margins: left=\the\dimen0, right=\the\trmargin}%
  \f@reachrow{%
    \ifnum\@row=2 \nobreak \fi
    \ifnum\@row=\tabler@ws \nobreak \fi
    \linec@unt=0
    \f@reachcol{\extractl@stcell \updatelinec@unt}%
    \trace{t}{outputting table (\the\tabler@ws, \the\linec@unt)}%
    \nth@line=0
    \@@LOOP \ifnum\nth@line<\linec@unt
      \advance\nth@line by 1
      \line{\hskip\dimen0
        \ifRTL
          \hfil\lochcaer@f{\csname \c@tprefix tc\the\@col :fill\endcsname\getnth@line}\hfil
        \else
          \hfil\f@reachcol{\getnth@line\csname \c@tprefix tc\the\@col :fill\endcsname}\hfil
        \fi
        \hskip\trmargin}%
      \ifnum\nth@line<\linec@unt \nobreak \fi
    \@@REPEAT
    \setbox\t@bleinserts=\vbox{\parfillskip=0pt\unvbox\t@bleinserts \global\setbox0=\lastbox}%
    \ifdim\ht0=1sp \else\unvbox0\fi
  }%
}
\newcount\linec@unt \newcount\lct@mp
\newdimen\sp@re \newdimen\exc@ss \newdimen\thr@shold

\def\@LOOP #1\@REPEAT{\gdef\@BODY{#1}\@ITERATE}
\def\@ITERATE{\@BODY \global\let\@NEXT\@ITERATE
 \else \global\let\@NEXT\relax \fi \@NEXT}
\let\@REPEAT\fi
% more copies of plain.tex loop macros (see also ptx-char-style)
\def\@@LOOP #1\@@REPEAT{\gdef\@@BODY{#1}\@@ITERATE}
\def\@@ITERATE{\@@BODY \global\let\@@NEXT\@@ITERATE
 \else \global\let\@@NEXT\relax \fi \@@NEXT}
\let\@@REPEAT\fi

\def\@@@LOOP #1\@@@REPEAT{\gdef\@@@BODY{#1}\@@@ITERATE}
\def\@@@ITERATE{\@@@BODY \global\let\@@@NEXT\@@@ITERATE
 \else \global\let\@@@NEXT\relax \fi \@@@NEXT}
\let\@@@REPEAT\fi

\def\s@tspansize{%
  \trace{t}{Spanning cell, orig hsize  \the\hsize}%
  \lct@mp=\csname span-\the\@row-\the\@col\endcsname
  \let\@@@nxt\@s@tspansize\@@@nxt
  \trace{t}{Spanning cell, hsize now \the\hsize}}%
\def\@s@tspansize{\ifnum \@col<\lct@mp 
    \advance\hsize by \csname colwidth-\the\lct@mp\endcsname
    \advance\lct@mp by -1
  \else\let\@@@nxt\relax\fi
  \@@@nxt}
  
  
\def\wr@pcolumn{\@row=1\setbox\c@rrentcolbox=\vbox{\parfillskip=0pt\unvbox\c@rrentcolbox \wr@pboxes}}
\def\wr@pboxes{% line-wrap a list of \hboxes
  \setbox0=\lastbox 
  \ifhbox0 {\advance\@row by 1 \wr@pboxes}\fi
  \setbox0=\vbox{\hsize=\c@rrentcolwidth
    \ifcsname span-\the\@row-\the\@col\endcsname \s@tspansize\fi
    \everypar={}\resetp@rstyle\noindent
    \leftskip=0pt\rightskip=0pt
    \ifRTL\dimen2=0pt \beginR\fi
    %\edef\tc@ll{tc\the\@col}%
    %\getp@ram{leftmargin}{\tc@ll}{\tc@ll}%
    %\ifx\p@ram\relax \else \advance \leftskip\p@ram \IndentUnit\fi
    %\getp@ram{firstindent}{\tc@ll}{\tc@ll}%
    %\ifx\p@ram\relax\else \dimen3=\p@ram \IndentUnit
    %  \ifRTL \dimen2=\dimen3 \else\ifdim\dimen3=0pt\else\kern \dimen3\fi\fi
    %\fi
    %\getp@ram{rightmargin}{\tc@ll}{\tc@ll}%
    %\ifx\p@ram\relax \else \advance \rightskip\p@ram \IndentUnit\fi
    \x@\let\x@\the@lign\csname align-\the\@row-\the\@col\endcsname
    \trace{t}{Pre(\c@tprefix \tc@ll, RTL\ifRTL true\else false\fi,TOCRTL\ifTOCRTL true\else false\fi) \c@tprefix cell(\the@lign)[\the\@row, \the\@col]: leftskip=\the\leftskip, rightskip=\the\rightskip, parfillskip=\the\parfillskip, width=\the\c@rrentcolwidth}%
    \edef\l@drsl{}\edef\l@drsr{}%
    \ifx\the@lign\cell@r
      \parfillskip=0pt
      \ifRTL \let\n@xt=\rightskip \edef\l@drsl{r}\else
        \skip0=\rightskip \rightskip=\leftskip \leftskip=\skip0
        \let\n@xt=\leftskip \edef\l@drsl{r}%
      \fi
    \else\ifx\the@lign\cell@c
      \parfillskip=0pt
      \let\n@xt=\rightskip \advance\leftskip by 0pt plus 1fil
     \else
      \parfillskip=0pt
      \ifRTL \skip0=\rightskip \rightskip=\leftskip \leftskip=\skip0
        \let\n@xt=\leftskip \edef\l@drsr{r}% \parfillskip=0pt
      \else\let\n@xt=\rightskip\trace{t}{n@xt=rightskip}\def\l@drsr{r}\fi
    \fi\fi
    \trace{t}{r=\l@drsr, l=\l@drsl, \c@tprefix =\c@ttoc}%
    \advance\n@xt by 0pt plus 1fil
    \edef\mytmp{\c@tprefix leaders-tc\l@drsr\the\@col}%
    \ifcsname \mytmp\endcsname \nobreak\csname \mytmp\endcsname\hfil\fi%
    \trace{t}{Post... \c@tprefix cell(\the@lign)[\the\@row, \the\@col]: leaders:\mytmp . leftskip=\the\leftskip, rightskip=\the\rightskip, parfillskip=\the\parfillskip, width=\the\c@rrentcolwidth}%
    %\hskip\skip1   % we already added that to the contents of the cell
    \unhbox0
    \edef\mytmp{\c@tprefix leaders-tc\l@drsl\the\@col}%
    \ifcsname \mytmp\endcsname
      \trace{t}{Remove space for leaders}%
      \parfillskip=0pt% \hskip\rightskip \rightskip=0pt
      \nobreak\csname \mytmp\endcsname \hfil
      \trace{t}{Final... \mytmp}\def\f@nish{\unskip\unskip}%
    \else \def\f@nish{}\fi
    \ifRTL\endR\fi\par\f@nish}% remove parfillskip and rightskip
  \wd0=\c@rrentcolwidth
  \box0}
\def\extractl@stcell{% delete last cell from box \c@rrentcolbox and return it in \c@rrentcell
  \setbox\c@rrentcolbox=\vbox{\parfillskip=0pt\unvbox\c@rrentcolbox \unskip\unpenalty
    \global\setbox\c@rrentcell=\lastbox}}
\def\updatelinec@unt{% count lines in \c@rrentcell, update \linec@unt if more than current value
  \global\lct@mp=0
  \setbox0=\vbox{\parfillskip=0pt\unvcopy\c@rrentcell \c@untlines}%
  \ifnum\lct@mp>\linec@unt \global\linec@unt=\lct@mp \fi}
\def\c@untlines{\setbox0=\lastbox
  \ifhbox0 \global\advance\lct@mp by 1 \unskip\unpenalty\c@untlines\fi}

\newif\ifh@lfrow
\def\getnth@line{% return content of line nth@line from box \c@rrentcell
  \getp@ram{verticalalign}{tc\the\@col}{tc\the\@col+\styst@k}%
  \ifx\p@ram\relax\let\p@ram\@lignTop\fi
  \lct@mp=0
  \h@lfrowfalse
  \bgroup %protect nth@line and count255
    \setbox0=\vbox{\parfillskip=0pt\unvcopy\c@rrentcell \c@untlines}% count lines into \lct@mp
    \count255=0
    \ifx\p@ram\@lignTop\else
      \count255=\numexpr \linec@unt - \lct@mp \relax
      \ifx\p@ram\@lignBot\else
        \ifx\p@ram\@lignGCent \else
          \ifodd\count255 \global\h@lfrowtrue\fi
        \fi
        \divide\count255 by 2
      \fi
    \fi
    \trace{t}{column \the\@col[\p@ram]. linecount:\the\linec@unt, lct@mp:\the\lct@mp, nthline:\the\nth@line, adj \the\count255 \ifh@lfrow .5\fi}%
    \advance \nth@line by -\count255
    \ifnum 1 = \ifnum\nth@line>\lct@mp 1 \else \ifnum\nth@line<1 1\else 0\fi\fi \global\setbox1=\hbox to \c@rrentcolwidth{\hfil}\else
      \setbox0=\vbox{\parfillskip=0pt\unvcopy\c@rrentcell
        \@@@LOOP \ifnum\nth@line<\lct@mp
          \setbox0=\lastbox \unskip\unpenalty
          \advance\lct@mp by -1 \@@@REPEAT
        \global\setbox1=\lastbox}%
    \fi
  \egroup
  \trace{t}{bls:\the\baselineskip,  width=\the\wd1}%
  \ifsp@ncols\wd1=\csname colwidth-\the\@col\endcsname \fi
  \ifh@lfrow\lower 0.5\baselineskip\fi
  \box1%
}
\newcount\nth@line

\def\f@reachcol#1{%
  \@col=0 \loop \ifnum\@col<\maxtablec@l
    \advance\@col by 1
    \x@\let\x@\c@rrentcolbox\csname tablecol-\the\@col\endcsname
    \x@\let\x@\c@rrentcolwidth\csname colwidth-\the\@col\endcsname
    \x@\let\x@\c@rrentcell\csname cell-\the\@col\endcsname
    #1\repeat}

\def\lochcaer@f#1{%
  \@col=\maxtablec@l \loop \ifnum\@col>0
    \x@\let\x@\c@rrentcolbox\csname tablecol-\the\@col\endcsname
    \x@\let\x@\c@rrentcolwidth\csname colwidth-\the\@col\endcsname
    \x@\let\x@\c@rrentcell\csname cell-\the\@col\endcsname
    #1%
    \advance\@col by -1
    \repeat}

\def\f@reachrow#1{%
  \@row=0 \@LOOP \ifnum\@row<\tabler@ws
    \advance\@row by 1
    #1\@REPEAT}

\newcount\tabler@ws
\newcount\maxtablec@l
\newcount\tablec@l
\newcount\@col \newcount\@row
\newcount\allocatedtablec@ls

\def\@llocnewc@l{%
  \advance\allocatedtablec@ls by 1
  \x@\n@wbox\csname tablecol-\the\allocatedtablec@ls\endcsname
  \x@\n@wbox\csname cell-\the\allocatedtablec@ls\endcsname
  \x@\n@wdimen\csname colwidth-\the\allocatedtablec@ls\endcsname
}
\let\n@wbox=\newbox
\let\n@wdimen=\newdimen
\bgroup \catcode`\^=7\obeylines
\gdef\ztablecategory{\bgroup\unc@tcodespecials\catcode`\ =10\obeylines\zt@blecategory}
\gdef\zt@blecategory #1^^M{\stripspace#1\end/ \end/\relax{#1}\xdef\tablecategory{\p@ram}\egroup\trace{t}{table category="\tablecategory"}}
\egroup
