% The original of this file provided by Vladan Majerech <maj at ktiml.mff.cuni.cz>
% Licence: 
% Use the license DEK used for TeX:):  No warranty, but no restrictions
%
%%%% WHOLE GAME %%%%
\newbox\QRcodebox
%
%
\def\DPI{600}%
\def\QRreset{\def\QRcontent{}}%
\def\QRdata{\QRreset\QRdataadd}%
\def\QRdataadd{\QRchars\QRdatacont}
\def\QRdatacont#1{\def\QRdatacontcont##1#1{\edef\QRcontent{\QRcontent##1}\plainchars}\QRdatacontcont}%
%^ want to allow active chars to allow UTF8 coding
\def\QRcode{\!\QRBIN\QRcontent\ENDQRBIN\ReedSolomon\QRSet\QRSelectMask\QRSetBox}%
\def\QRminVersion{1}%
\def\QRmaxVersion{40}%
% \QRdimen=4cm to choose the box side
% interface \QRdataadd x<data not containing x>x repeated several times
%\QRCode uses gathered data and fills QRcodebox
%\QRreset resets the data \QRdata calls reset first (\QRdata.. does the same as \QRreset)
%\QRdimen allows setting size of QRcodebox including the quiet zone
%\DPI is used to find optimal granularity according to printer resolution
%\QRminVersion allows prohibiting smaller versions
%\QRmaxVersion allows prohibiting higher versions Issues \errmessage{No room in versions till ...}
%\QRminVerion and \QRmaxVersion is used in default \freeVersion mode
%\fixVersion{5} is an alternative to use given version number even when smaller version is sufficient
%\QRSetMinECLevel allows to force at least that error correction. The EC level is chosen after the version, but the version choice is constrained by the min level
%\QRSetECLevel allows to fix level of error correction
%
% The code is optimized for using big collection of similar sized QR codes at one run ... after a QRcode version is used,
% it's template is remembered for further use. Masking is done on the fly. 
% (It was desined first for generating QRpayments for a lot of clients (versions 4 and 5 were sufficient), 
% but full set of versions is implemented now).
% In default setting it chooses minimal sized QR codes with maximal possible error level for the size.
% Using even small QRcodes in emTeX fails due to POOL SIZE
% all versions of QRcodes work for me in mikTeX, but printing several (big) codes on the same page exceeds MAIN MEMORY size
% this is why I have not implemented chains in the planned way ... 
% as I expect putting several big codes into one box would exceed the MAIN memory
% using chains for smaller codes could be worth trying so it still remains in TODO list ... with small priority
%
% Author: Vladan Majerech maj@ktiml.mff.cuni.cz
%
% There are example formats here after \endinput
% usage for czech mecards require UTF8 encodding conversion (from 1250 codepage), 
% extension for other alphabets would be similar
%%%% WHERE TO CONTINUE? NOT DONE YET! %%%%
% Input the text file instead
%\QRmaxChainLen allows prohibiting QRCode chains
% Planned priorities to select Version, error correction and chain len is following:
% 1) minimize chain len as possible (no chains implemented yet)
% 2) chose minimal version possible to be capable to code the data, use same version in the chain.
% 3) chose maximal correction level X to be capable to code the data
% 4) if using correction level X+1 for the current part allows coding rest of the message in the remaining parts of the chain using correction level X, use correction level X+1 for this part, otherwise use correction level X.
% When there is not enough capacity the error should be issued
% If more than one box is used, the box \QRcodebox would be composed of all the boxes. They could be separated by \lastbox.
%\QRchainLen is useful in that case
%
%%%%%% GLOBAL %%%%%%
\let\!=\expandafter
\newif\iftest
\let\orioplus\oplus
\def\oplus{\nobreak\orioplus\nobreak\discretionary{}{\hbox{$\orioplus$}}{}}%
\newcount\QRnumA\newcount\QRnumB\newcount\QRnumC\newcount\QRnumD\newcount\QRnumE\newcount\QRnumF\newcount\QRnumG
\def\QRempty{}%
\def\QRspace{ }%
\def\QRtypeN{N}%
\def\QRtypeA{A}%
\def\QRtypeB{B}%
\def\gobbleone#1{}%
\def\gobblesecond#1#2{#1}%
\def\doubleToken#1{#1#1}%
%\def\gobbletwo#1#2{}
%
%%%%% QR sizes %%%%%
\def\QRdefBlocks#1#2#3#4#5#6#7#8#9{%Sizes defined with a random choice ... table needed
\QRnumA=#1 %version
\QRnumB=4 \multiply\QRnumB by \QRnumA\advance\QRnumB by 17 %side
\!\edef\csname QRside<\romannumeral#1>\endcsname{\the\QRnumB\QRspace}%
\QRnumC=\QRnumB\multiply\QRnumC by \QRnumC %area
\ifnum\QRnumA>1
 \QRnumD=\QRnumA\divide\QRnumD by 7 
 \QRnumF=\QRnumD %# alignments between FINDERS on one side
 \advance\QRnumD by 1 %#alignments per side except on timing
 \!\edef\csname QRSideAligns<\romannumeral#1>\endcsname{\the\QRnumD\QRspace}%
 \QRnumE=\QRnumD\multiply\QRnumE by \QRnumE\multiply\QRnumE by 25 %area of alignments except those on timing
 \multiply\QRnumF by 40 \advance\QRnumE by \QRnumF% area of alignments including alignments on timing, but excluding timing
 %\message{Sizes version \the\QRnumA, side \the\QRnumB, area \the\QRnumC, alignside \the\QRnumD, alignsarea \the\QRnumE, alignsontimersarea \the\QRnumF}%
\else
 \QRnumD=0 %
 \!\edef\csname QRSideAligns<\romannumeral#1>\endcsname{\the\QRnumD\QRspace}%
\fi
\QRnumF=\QRnumC\advance\QRnumF by -192 %data area ... finders excluded so far
\advance\QRnumF by -\QRnumE% alignments except timing excluded
\QRnumE=\QRnumB\advance\QRnumE by -16\multiply \QRnumE by 2 %timing pattern area
%\message{;timing \the\QRnumE,area without finders and aligns \the\QRnumF}%
\advance\QRnumF by -\QRnumE% timing area excluded as well
%\message{;area without finders, aligns and timing\the\QRnumF}%
\ifnum\QRnumA>6
 \advance\QRnumF by -36 % version encodding area excluded as well
 %\message{;data+ECmask info area\the\QRnumF}%
\fi
\advance\QRnumF by -31 % encodding mask and dummy bit excluded as well
%\message{;data area \the\QRnumF}%
%now \QRnumF contains area capable to hold info
\QRnumE=\QRnumF
\divide\QRnumE by 8 %number of bytes available for data and error correction
\multiply\QRnumE by 8 %number of bits of the available bytes
\advance\QRnumF by -\QRnumE % bits reminder
\!\edef\csname QRPadBits<\romannumeral#1>\endcsname{\the\QRnumF\QRspace}%
\divide\QRnumE by 8 %number of bytes available for data and error correction
%\message{;byte size\the\QRnumE,padding bits\the\QRnumF}%
\!\def\csname QRBlocks<\romannumeral#1>L\endcsname{#2 }%
\!\def\csname QREC<\romannumeral#1>L\endcsname{#3 }%
\QRnumF#2 \multiply\QRnumF by -#3 %totalECbytes
\advance\QRnumF by \QRnumE %total data bytes
\multiply\QRnumF by 8 %total data bits
\!\edef\csname QRBitLimit<\romannumeral#1>L\endcsname{\the\QRnumF\QRspace}%
\!\def\csname QRBlocks<\romannumeral#1>M\endcsname{#4 }%
\!\def\csname QREC<\romannumeral#1>M\endcsname{#5 }%
\QRnumF#4 \multiply\QRnumF by -#5 %totalECbytes
\advance\QRnumF by \QRnumE %total data bytes
\multiply\QRnumF by 8 %total data bits
\!\edef\csname QRBitLimit<\romannumeral#1>M\endcsname{\the\QRnumF\QRspace}%
\!\def\csname QRBlocks<\romannumeral#1>Q\endcsname{#6 }%
\!\def\csname QREC<\romannumeral#1>Q\endcsname{#7 }%
\QRnumF#6 \multiply\QRnumF by -#7 %totalECbytes
\advance\QRnumF by \QRnumE %total data bytes
\multiply\QRnumF by 8 %total data bits
\!\edef\csname QRBitLimit<\romannumeral#1>Q\endcsname{\the\QRnumF\QRspace}%
\!\def\csname QRBlocks<\romannumeral#1>H\endcsname{#8 }%
\!\def\csname QREC<\romannumeral#1>H\endcsname{#9 }%
\QRnumF#8 \multiply\QRnumF by -#9 %totalECbytes
\advance\QRnumF by \QRnumE %total data bytes
\multiply\QRnumF by 8 %total data bits
\!\edef\csname QRBitLimit<\romannumeral#1>H\endcsname{\the\QRnumF\QRspace}%
}%
\def\QRSetMinECLevel#1{\QRnumA=40
\loop \!\edef\csname QRBitLimit<\romannumeral\QRnumA>\endcsname{\!\noexpand\csname QRBitLimit<\romannumeral\QRnumA>#1\endcsname}%
\advance\QRnumA by -1 \ifnum\QRnumA>0\repeat
\let\chooseECLevel=\chooseMaxECLevel
}%
\def\QRSetECLevel#1{\QRSetMinECLevel{#1}\def\EClevel{#1}\let\chooseECLevel=\relax}%
\QRSetMinECLevel{L}%
\QRdefBlocks{1}{1}{7}{1}{10}{1}{13}{1}{17}%
\QRdefBlocks{2}{1}{10}{1}{16}{1}{22}{1}{28}%
\QRdefBlocks{3}{1}{15}{1}{26}{2}{18}{2}{22}%
\QRdefBlocks{4}{1}{20}{2}{18}{2}{26}{4}{16}%
\QRdefBlocks{5}{1}{26}{2}{24}{4}{18}{4}{22}%
\QRdefBlocks{6}{2}{18}{4}{16}{4}{24}{4}{28}%
\QRdefBlocks{7}{2}{20}{4}{18}{6}{18}{5}{26}%
\QRdefBlocks{8}{2}{24}{4}{22}{6}{22}{6}{26}%
\QRdefBlocks{9}{2}{30}{5}{22}{8}{20}{8}{24}%
\QRdefBlocks{10}{4}{18}{5}{26}{8}{24}{8}{28}%
\QRdefBlocks{11}{4}{20}{5}{30}{8}{28}{11}{24}%
\QRdefBlocks{12}{4}{24}{8}{22}{10}{26}{11}{28}%
\QRdefBlocks{13}{4}{26}{9}{22}{12}{24}{16}{22}%
\QRdefBlocks{14}{4}{30}{9}{24}{16}{20}{16}{24}%
\QRdefBlocks{15}{6}{22}{10}{24}{12}{30}{18}{24}%
\QRdefBlocks{16}{6}{24}{10}{28}{17}{24}{16}{30}%
\QRdefBlocks{17}{6}{28}{11}{28}{16}{28}{19}{28}%
\QRdefBlocks{18}{6}{30}{13}{26}{18}{28}{21}{28}%
\QRdefBlocks{19}{7}{28}{14}{26}{21}{26}{25}{26}%
\QRdefBlocks{20}{8}{28}{16}{26}{20}{30}{25}{28}%
\QRdefBlocks{21}{8}{28}{17}{26}{23}{28}{25}{30}%
\QRdefBlocks{22}{9}{28}{17}{28}{23}{30}{34}{24}%
\QRdefBlocks{23}{9}{30}{18}{28}{25}{30}{30}{30}%
\QRdefBlocks{24}{10}{30}{20}{28}{27}{30}{32}{30}%
\QRdefBlocks{25}{12}{26}{21}{28}{29}{30}{35}{30}%
\QRdefBlocks{26}{12}{28}{23}{28}{34}{28}{37}{30}%
\QRdefBlocks{27}{12}{30}{25}{28}{34}{30}{40}{30}%
\QRdefBlocks{28}{13}{30}{26}{28}{35}{30}{42}{30}%
\QRdefBlocks{29}{14}{30}{28}{28}{38}{30}{45}{30}%
\QRdefBlocks{30}{15}{30}{29}{28}{40}{30}{48}{30}%
\QRdefBlocks{31}{16}{30}{31}{28}{43}{30}{51}{30}%
\QRdefBlocks{32}{17}{30}{33}{28}{45}{30}{54}{30}%
\QRdefBlocks{33}{18}{30}{35}{28}{48}{30}{57}{30}%
\QRdefBlocks{34}{19}{30}{37}{28}{51}{30}{60}{30}%
\QRdefBlocks{35}{19}{30}{38}{28}{53}{30}{63}{30}%
\QRdefBlocks{36}{20}{30}{40}{28}{56}{30}{66}{30}%
\QRdefBlocks{37}{21}{30}{43}{28}{59}{30}{70}{30}%
\QRdefBlocks{38}{22}{30}{45}{28}{62}{30}{74}{30}%
\QRdefBlocks{39}{24}{30}{47}{28}{65}{30}{77}{30}%
\QRdefBlocks{40}{25}{30}{49}{28}{68}{30}{81}{30}%
%
%%%%% DATA LEVEL %%%%%
\def\QRBIN#1\ENDQRBIN{% PATTERN OPTIMIZATION first
\QRswitchCostsLow\QRDiag#1\ENDQRDiag
\ifnum\QRnumF>\QRLowLimit \QRswitchCostsMed\QRDiag#1\ENDQRDiag\fi
\ifnum\QRnumF>\QRMedLimit \QRswitchCostsHigh\QRDiag#1\ENDQRDiag\fi
%\message{\the\QRnumF(\QRSchema)}%
\binaryReset\PACKcontent#1\PAD}%
%
%%%%% DATA TRANSLATION %%%%%
%%%%% TABLE 1250->UTF8 %%%%%
\def\UTFlist{}%
\def\UTF#1#2{}%\!\def\!\UTFlist\!{\UTFlist\UTFdo#1}\bgroup\uccode`~=`#1\uppercase{\egroup\def~}{#2}}%
\UTF{á}{^^c3^^a1}\UTF{ä}{^^c3^^a4}\UTF{è}{^^c4^^8d}\UTF{ï}{^^c4^^8f}\UTF{é}{^^c3^^a9}%
\UTF{ì}{^^c4^^9b}\UTF{ë}{^^c3^^ab}\UTF{í}{^^c3^^ad}\UTF{å}{^^c4^^ba}\UTF{¾}{^^c4^^be}%
\UTF{ò}{^^c5^^88}\UTF{ó}{^^c3^^b3}\UTF{ô}{^^c3^^b4}\UTF{ö}{^^c3^^b6}\UTF{à}{^^c5^^95}%
\UTF{ø}{^^c5^^99}\UTF{š}{^^c5^^a1}\UTF{}{^^c5^^a5}\UTF{ú}{^^c3^^ba}\UTF{ù}{^^c5^^af}%
\UTF{ü}{^^c3^^bc}\UTF{ý}{^^c3^^bd}\UTF{ž}{^^c5^^be}%
\UTF{Á}{^^c3^^81}\UTF{Ä}{^^c3^^84}\UTF{È}{^^c4^^8c}\UTF{Ï}{^^c4^^8e}\UTF{É}{^^c3^^89}%
\UTF{Ì}{^^c4^^9a}\UTF{Ë}{^^c3^^8b}\UTF{Í}{^^c3^^8d}\UTF{Å}{^^c4^^b9}\UTF{¼}{^^c4^^bd}%
\UTF{Ò}{^^c5^^87}\UTF{Ó}{^^c3^^93}\UTF{Ô}{^^c3^^94}\UTF{Ö}{^^c3^^96}\UTF{À}{^^c5^^94}%
\UTF{Ø}{^^c5^^98}\UTF{Š}{^^c5^^a0}\UTF{}{^^c5^^a4}\UTF{Ú}{^^c3^^9a}\UTF{Ù}{^^c5^^ae}%
\UTF{Ü}{^^c3^^9c}\UTF{Ý}{^^c3^^9d}\UTF{Ž}{^^c5^^bd}%
%\UTF{á}{a}\UTF{ä}{a}\UTF{è}{c}\UTF{ï}{d}\UTF{é}{e}%
%\UTF{ì}{e}\UTF{ë}{e}\UTF{í}{i}\UTF{å}{l}\UTF{¾}{l}%
%\UTF{ò}{n}\UTF{ó}{o}\UTF{ô}{o}\UTF{ö}{o}\UTF{à}{r}%
%\UTF{ø}{r}\UTF{š}{s}\UTF{}{t}\UTF{ú}{u}\UTF{ù}{u}%
%\UTF{ü}{u}\UTF{ý}{y}\UTF{ž}{z}%
%\UTF{Á}{A}\UTF{Ä}{A}\UTF{È}{C}\UTF{Ï}{D}\UTF{É}{E}%
%\UTF{Ì}{E}\UTF{Ë}{E}\UTF{Í}{I}\UTF{Å}{L}\UTF{¼}{L}%
%\UTF{Ò}{N}\UTF{Ó}{O}\UTF{Ô}{O}\UTF{Ö}{O}\UTF{À}{R}%
%\UTF{Ø}{R}\UTF{Š}{S}\UTF{}{T}\UTF{Ú}{U}\UTF{Ù}{U}%
%\UTF{Ü}{U}\UTF{Ý}{Y}\UTF{Ž}{Z}%
\def\CATlist{}%
\let\CATdo\noexpand
\def\CAT#1{\!\!\!\def\!\!\!\CATlist\!\!\!{\!\CATlist\!\CATdo\!#1\!{\the\catcode`#1}}}
\def\makeactive#1{\catcode`#1=13 }%
\def\makeother#1{\catcode`#1=12 }%
\def\CATother#1#2{\catcode`#1=12 }%
\def\CATori#1#2{\catcode`#1=#2 }%
%\CAT\%\CAT\$\CAT\ \CAT\\
\let\do\CAT\dospecials
%\show\CATlist
%\def\QRchars{\catcode`\%=12 \catcode`\$=12 \catcode`\ =12 \let\UTFdo\makeactive\UTFlist \catcode`\\=12 }%
\def\QRchars{\let\CATdo=\CATother\CATlist\let\UTFdo\makeactive\UTFlist}%
%\def\plainchars{\catcode`\\=0 \catcode`\ =10\relax\catcode`\%=14 \catcode`\$=3 \let\UTFdo\makeother\UTFlist}%
\def\plainchars{\let\CATdo=\CATori\CATlist\let\UTFdo\makeother\UTFlist}%
{\!\QRchars\!\catcode\!`\!\}\!=\!2\! \!\gdef\!\ALPHANUMERIC{0123456789ABCDEFGHIJKLMNOPQRSTUVWXYZ $%*+-./:}}%
%%%%% MODE CATEGORIES %%%%%
\def\NUMERIC{0123456789}%
\def\isALPHANUMERIC#1{\def\tst##1#1##2\endtest{\if @##2@\testfalse\else\testtrue\fi}%
\!\tst\ALPHANUMERIC#1\endtest}%
\def\isNUMERIC#1{\def\tst##1#1##2\endtest{\if @##2@\testfalse\else\testtrue\fi}\!\tst\NUMERIC#1\endtest}%
\def\defQRtype#1{\isNUMERIC#1\iftest\def\QRtype{N}\else\isALPHANUMERIC#1\iftest\def\QRtype{A}\else\def\QRtype{B}\fi\fi}%
%%%%% CHAR TO NUMBER ACCORDING MODE %%%%%
\def\BINNUM#1{\QRnumA=`#1}%
\def\ALPHANUM#1{%could be used both for ALPHANUMERIC and NUMERIC
\let\ALPHANUMnext\MATCHALPHA\QRnumA=0 \!\ALPHANUMnext\!#1\ALPHANUMERIC\ENDALPHA}%
\def\MATCHALPHA#1#2#3\ENDALPHA{\if#1#2\let\ALPHANUMnext\MATCHEDALPHA\else\advance\QRnumA by 1 \fi
\ALPHANUMnext#1#3\ENDALPHA}%
\def\MATCHEDALPHA#1#2\ENDALPHA{}%
%
%%%%% PACKING PATTERN OPTIMIZATION %%%%%
\def\QRBswitchCostLow{12 }\def\QRBswitchCostMed{20 }\def\QRBswitchCostHigh{20 }%
\def\QRAswitchCostLow{13 }\def\QRAswitchCostMed{15 }\def\QRAswitchCostHigh{17 }%
\def\QRNswitchCostLow{14 }\def\QRNswitchCostMed{16 }\def\QRNswitchCostHigh{18 }%
\def\QRswitchCostsLow{\let\QRBswitchCost=\QRBswitchCostLow\let\QRAswitchCost=\QRAswitchCostLow\let\QRNswitchCost=\QRNswitchCostLow}%1-9
\def\QRswitchCostsMed{\let\QRBswitchCost=\QRBswitchCostMed\let\QRAswitchCost=\QRAswitchCostMed\let\QRNswitchCost=\QRNswitchCostMed}%10-26
\def\QRswitchCostsHigh{\let\QRBswitchCost=\QRBswitchCostHigh\let\QRAswitchCost=\QRAswitchCostHigh\let\QRNswitchCost=\QRNswitchCostHigh}%27-40
\def\QRLowLimit{\csname QRBitLimit<ix>\endcsname}\def\QRMedLimit{\csname QRBitLimit<xxvi>\endcsname}% def no let as the macros could change by \QRSetMinECLevel
\def\QRinfty{29649 }%
\def\QRDiag#1\ENDQRDiag{%\message{Diag >#1< Limits \QRLowLimit \QRMedLimit}%
% should be changed if chosen version is higher than 10/27 and rerun again ... 
%A:number of chars of the last type, 
%B:number of not yet counted chars when ending by type A, 
%C:number of not yet counted chars where ending by type B,
%D:length, when ending by type N including the last sequence \infty is represented by one more than version40L capacity \QRinfty
%E:length, when ending by type A excluding the last B chars, \infty is represented by \QRinfty again
%F:length, when ending by type B excluding last C chars
%Returns F the optimal length and QRSchema the optimal schema
%subresults QRSchemaB, QRSchemaA, QRSchemaN representing optimal schemas ending with the appropriate type (excluding the last sequence in the B, A cases)
\QRnumA=0 \QRnumB=\QRnumA \QRnumC=\QRnumA 
\def\QRSchemaB{}\def\QRSchemaA{}\def\QRSchemaN{}%
\QRnumD=\QRNswitchCost\QRnumE=\QRAswitchCost\QRnumF=\QRBswitchCost
\let\QRDIAGnext=\DIAGcontent\DIAGcontent#1\ENDDIAG}%
\def\DIAGcontent#1#2\ENDDIAG{\defQRtype#1\ifnum\QRnumA>0 \ifx\prevQRtype\QRtype\else \QRtypeChange\fi\fi\let\prevQRtype=\QRtype\advance\QRnumA by 1 \def\test{#2}\ifx\test\QRempty\let\QRDIAGnext=\QRDIAGend\fi\QRDIAGnext#2\ENDDIAG}%
\def\QRDIAGend\ENDDIAG{\QRtypeChange
%\message{\prevQRtype\the\QRnumA,\the\QRnumD\QRSchemaN,\the\QRnumE[\the\QRnumB]\QRSchemaA,\the\QRnumF[\the\QRnumC]\QRSchemaB}%
% argmin 1) F+C*8, 2) E+\lfloor(11B+1)/2 \rfloor, 3) D
% 1) \edef\QRSchema{\QRSchemaB B\the\QRnumC} F:=F+C*8
% 2) \edef\QRSchema{\QRSchemaA A\the\QRnumB} F:=E+\lfloor(11B+1)/2 \rfloor
% 3) \let\QRSchema=\QRSchemaN F:=D
 \QRnumG=\QRnumC\multiply\QRnumG by 8 \advance\QRnumF by\QRnumG
 \QRnumG=\QRnumB\multiply\QRnumG by 11 \advance\QRnumG by 1 \divide\QRnumG by 2 \advance\QRnumE by \QRnumG
 \ifnum\QRnumE<\QRnumD
  \ifnum\QRnumE<\QRnumF
   \edef\QRSchema{\QRSchemaA A\the\QRnumB}\QRnumF=\QRnumE
  \else
   \edef\QRSchema{\QRSchemaB B\the\QRnumC}%
  \fi
 \else
  \ifnum\QRnumD<\QRnumF
   \let\QRSchema=\QRSchemaN\QRnumF=\QRnumD
  \else
   \edef\QRSchema{\QRSchemaB B\the\QRnumC}%
  \fi
 \fi
}%
\def\QRtypeChange{%
%\message{\prevQRtype\the\QRnumA,\the\QRnumD\QRSchemaN,\the\QRnumE[\the\QRnumB]\QRSchemaA,\the\QRnumF[\the\QRnumC]\QRSchemaB}%
\ifx\prevQRtype\QRtypeB %B switch
% argmin 1) F+C*8, 2) E+\lfloor(11B+1)/2 \rfloor+\QRBswitchCost, 3) D+\QRBswitchCost according the argmin:
% 1) F remains, C+=A, \QRSchemaB remains
% 2) F:=E+\lfloor(11B+1)/2 \rfloor+\QRBswitchCost, C=A, \edef\QRSchemaB{\QRSchemaA A\the\QRnumB}
% 3) F:=D+\QRBswitchCost, C=A \let\QRSchemaB=\QRSchemaN
% E=\QRinfty, D=\QRinfty
 \QRnumG=\QRnumC\multiply\QRnumG by 8 \advance\QRnumG by\QRnumF\edef\exprB{\the\QRnumG}%
 \QRnumG=\QRnumB\multiply\QRnumG by 11 \advance\QRnumG by 1 \divide\QRnumG by 2 \advance\QRnumG by \QRnumE
 \ifnum\QRnumG<\QRnumD
  \advance\QRnumG by \QRBswitchCost
  \ifnum\exprB>\QRnumG
   \QRnumF=\QRnumG\QRnumC=\QRnumA\edef\QRSchemaB{\QRSchemaA A\the\QRnumB}%
  \else
   \advance\QRnumC by \QRnumA
  \fi
 \else
  \QRnumG=\QRnumD
  \advance\QRnumG by \QRBswitchCost
  \ifnum\exprB>\QRnumG
   \QRnumF=\QRnumG\QRnumC=\QRnumA\let\QRSchemaB=\QRSchemaN
  \else
   \advance\QRnumC by \QRnumA
  \fi 
 \fi
 \QRnumE=\QRinfty\QRnumD=\QRinfty
\else\ifx\prevQRtype\QRtypeA %A switch
% argmin 1) F+(C+A)*8, 2) E+\lfloor(11(B+A)+1)/2 \rfloor+\QRBswitchCost, 3) D+\QRBswitchCost+A*8 according the argmin:
% 1) futF=F remains, \def\futC{\advance\QRnumC by \QRnumA}, \let\futQRSchemaB=\QRSchemaB
% 2) futF:=E+\lfloor(11(B+A)+1)/2 \rfloor+\QRBswitchCost, \edef\futQRSchemaB{\QRSchemaA A\the(\QRnumB+\QRnumA) counted in G}  \def\futC{\QRnumC=0 }
% 3) futF:=D+\QRBswitchCost, \def\futC{\QRnumC=\QRnumA} \let\futQRSchemaB=\QRSchemaN
% argmin 1) F+C*8+\QRAswitchCost, 2) E+\lfloor(11B+1)/2 \rfloor, 3) D+\QRAswitchCost according the argmin:
% 1) E:=F+C*8+\QRAswitchCost, B=A \edef\QRSchemaA{\QRSchemaB B\the\QRnumC}
% 2) E remains, B=B+A \QRSchemaA remains
% 3) E:=D+\QRBswitchCost, B=A \let\QRSchemaA=\QRSchemaN
% \let\QRSchemaB=\futQRSchemaB\futC
% D=\QRinfty
 \QRnumG=\QRnumC\advance\QRnumG by \QRnumA\multiply\QRnumG by 8 \advance\QRnumG by\QRnumF\edef\exprB{\the\QRnumG}%
 \QRnumG=\QRnumB\advance\QRnumG by \QRnumA\multiply\QRnumG by 11 \advance\QRnumG by 1 \divide\QRnumG by 2 \advance\QRnumG by \QRnumE
 \edef\exprA{\the\QRnumG\QRspace}%
 \QRnumG=\QRnumA\multiply\QRnumG by 8 \advance\QRnumG by\QRnumD
 \ifnum\exprA<\QRnumG
  \QRnumG=\exprA
  \advance\QRnumG by \QRBswitchCost
  \ifnum\exprB>\QRnumG
   \edef\futF{\QRnumF=\the\QRnumG\QRspace}\QRnumG=\QRnumB\advance\QRnumG by \QRnumA\edef\futQRSchemaB{\QRSchemaA A\the\QRnumG}\def\futC{\QRnumC=0 }%
  \else
   \let\futF=\relax\def\futC{\advance\QRnumC by \QRnumA}\let\futQRSchemaB=\QRSchemaB
  \fi
 \else
  \advance\QRnumG by \QRBswitchCost
  \ifnum\exprB>\QRnumG
   \QRnumG=\QRnumD\advance\QRnumG by \QRBswitchCost
   \edef\futF{\QRnumF=\the\QRnumG\QRspace}\def\futC{\QRnumC=\QRnumA}\let\futQRSchemaB=\QRSchemaN
  \else
   \let\futF=\relax\def\futC{\advance\QRnumC by \QRnumA}\let\futQRSchemaB=\QRSchemaB
  \fi
 \fi
 \QRnumG=\QRnumB\multiply\QRnumG by 11 \advance\QRnumG by 1 \divide\QRnumG by 2 \advance\QRnumG by \QRnumE\edef\exprA{\the\QRnumG}%
 \QRnumG=\QRnumC\multiply\QRnumG by 8 \advance\QRnumG by \QRnumF
 \ifnum\QRnumG<\QRnumD
  \advance\QRnumG by \QRAswitchCost
  \ifnum\exprA>\QRnumG
   \QRnumB=\QRnumA\edef\QRSchemaA{\QRSchemaB B\the\QRnumC}\QRnumE=\QRnumG
  \else
   \advance\QRnumB by \QRnumA
  \fi
 \else
  \QRnumG=\QRnumD
  \advance\QRnumG by \QRAswitchCost
  \ifnum\exprA>\QRnumG
   \QRnumB=\QRnumA\let\QRSchemaA=\QRSchemaN\QRnumE=\QRnumG
  \else
   \advance\QRnumB by \QRnumA
  \fi
 \fi
 \let\QRSchemaB=\futQRSchemaB\futF\futC
 \QRnumD=\QRinfty
\else %N switch
% argmin 1) F+(C+A)*8, 2) E+\lfloor(11(B+A)+1)/2 \rfloor+\QRBswitchCost  according the argmin:
% 1) futF:=F remains, \def\futC{\advance\QRnumC by \QRnumA}, \let\futQRSchemaB=\QRSchemaB
% 2) futF:=E+\lfloor(11(B+A)+1)/2 \rfloor+\QRBswitchCost, \edef\futQRSchemaB{\QRSchemaA A\the(\QRnumB+\QRnumA) counted in G}  \def\futC{\QRnumC=0 }
% argmin 1) F+C*8+\QRAswitchCost, 2) E+\lfloor(11B+1)/2 \rfloor according the argmin:
% 1) futE:=F+C*8+\QRAswitchCost, \def\futB{\QRnumB=\QRnumA}, \edef\futQRSchemaA{\QRSchemaB B\the\QRnumC}
% 2) futE:=E remains, \def\futB{\advance\QRnumB by \QRnumA}, \futQRSchemaA=\QRSchemaA
% argmin 1) F+C*8, 2) E+\lfloor(11B+1)/2 \rfloor 3) D according the argmin:
% 1)D:=F+C*8+\QRNswitchCost+\lfloor(10A+2)/3 \rfloor, \edef\QRSchemaN{\QRSchemaB B\the\QRnumC N\the\QRnumA}
% 2)D:=E+\lfloor(11B+1)/2 \rfloor)+\QRNswitchCost+\lfloor(10A+2)/3 \rfloor, \edef\QRSchemaN{\QRSchemaA A\the\QRnumB N\the\QRnumA}
% 3)D:=D+\lfloor(10A+2)/3 \rfloor for the case of start of the message \edef\QRSchemaN{N\the\QRnumA}
% \let\QRSchemaB=\futQRSchemaB\futC
% \let\QRSchemaA=\futQRSchemaA\futB
 \QRnumG=\QRnumA\advance\QRnumG by \QRnumC\multiply\QRnumG by 8 \advance\QRnumG by \QRnumF\edef\exprB{\the\QRnumG\QRspace}%
 \QRnumG=\QRnumA\advance\QRnumG by \QRnumB\multiply\QRnumG by 11 \advance\QRnumG by 1 \divide\QRnumG by 2 \advance\QRnumG by \QRnumE\edef\exprA{\the\QRnumG\QRspace}% 
 \advance\QRnumG by \QRBswitchCost
 \ifnum\exprB>\QRnumG
  \edef\futF{\QRnumF=\the\QRnumG\QRspace}\QRnumG=\QRnumB\advance\QRnumG by \QRnumA
  \edef\futQRSchemaB{\QRSchemaA A\the\QRnumG}\def\futC{\QRnumC=0 }%
 \else
  \let\futF=\relax\def\futC{\advance\QRnumC by \QRnumA}\let\futQRSchemaB=\QRSchemaB
 \fi
 \QRnumG=\QRnumB\multiply\QRnumG by 11 \advance\QRnumG by 1 \divide\QRnumG by 2 \advance\QRnumG by \QRnumE\edef\exprA{\the\QRnumG\QRspace}% 
 \QRnumG=\QRnumC\multiply\QRnumG by 8 \advance\QRnumG by \QRnumF\edef\exprB{\the\QRnumG\QRspace}%
 \advance\QRnumG by \QRAswitchCost
 \ifnum\exprA>\QRnumG
  \edef\futE{\QRnumE=\the\QRnumG\QRspace}\def\futB{\QRnumB=\QRnumA}\edef\futQRSchemaA{\QRSchemaB B\the\QRnumC}%
 \else
  \let\futE=\relax\def\futB{\advance\QRnumB by \QRnumA}\let\futQRSchemaA=\QRSchemaA
 \fi
 \QRnumG=\QRnumD
 \QRnumD=\QRnumA\multiply\QRnumD by 10 \advance\QRnumD by 2 \divide\QRnumD by 3 \advance\QRnumD by \QRNswitchCost
 \ifnum\QRnumG=\QRNswitchCost
  \edef\QRSchemaN{N\the\QRnumA}%
 \else
  \ifnum\exprA<\exprB
   \edef\QRSchemaN{\QRSchemaA A\the\QRnumB N\the\QRnumA}\advance\QRnumD by \exprA
  \else
   \edef\QRSchemaN{\QRSchemaB B\the\QRnumC N\the\QRnumA}\advance\QRnumD by \exprB
  \fi
 \fi 
 \let\QRSchemaB=\futQRSchemaB\futF\futC
 \let\QRSchemaA=\futQRSchemaA\futE\futB
\fi\fi 
\QRnumA=0
}%
%%%%% PACKING %%%%%
\def\binaryReset{\QRnumC=0 \def\binary{}}%length of binary is maintained in \QRnumC 
%
\def\NUMBIN{%\message{NUMBIN [\the\QRnumC]&\the\QRnumA(\the\QRnumB)}%
%writes \QRnumA using \QRnumB binary digits to the end of \binary,
% increases \QRnumC by the number of digits added to \binary
\advance\QRnumC by \QRnumB
\ifnum\QRnumB>0 \let\NUMBINnext\NUMBINNEXTDIGIT\NUMBINnext\ENDNUMBIN\fi}%
%never more than 16 bits :) ... the recursion would not be a tail one
\def\NUMBINNEXTDIGIT{\ifodd\QRnumA\def\NUMBINnextdigit{\1}\else\def\NUMBINnextdigit{\0}\fi
\divide\QRnumA by 2 \advance\QRnumB by -1 \ifnum\QRnumB=0 \let\NUMBINnext\NUMBINEND\fi
\!\NUMBINnext\NUMBINnextdigit}%
\def\NUMBINEND#1\ENDNUMBIN{\!\def\!\binary\!{\binary#1}}%
%
\def\showOne{1\penalty0 }\def\showZero{0\penalty0 }%
\def\showing{\let\0\showZero\let\1\showOne}%
\def\binaryShow{\let\1=\showOne\let\0=\showZero (\the\QRnumC)\binary}%

\def\PACKcontent{%\message{packing with schema \QRSchema}%
\ifx\QRSchema\QRempty\ifnum\QRnumC=\QRnumF\else\errmessage{Number of bits does not correspond to the computed value}\fi\else\!\takeFirstQRSchema\QRSchema\relax\!\PackQRTypeBlock\fi}%TODO
\def\takeFirstQRSchema#1{\def\QRtype{#1}\afterassignment\takeFirstQRSchemaCont\QRnumD=}%
\def\takeFirstQRSchemaCont#1\relax{\def\QRSchema{#1}%
%\message{\QRtype\the\QRnumD, remains >\QRSchema<}%
}%
\def\PackQRTypeBlock{%\message{PackQRTypeBlock \QRtype\the\QRnumD}%
\ifx\QRtype\QRtypeB\QRnumA=4 \QRnumE=\QRBswitchCost
\else
 \ifx\QRtype\QRtypeA\QRnumA=2 \QRnumE=\QRAswitchCost
 \else \QRnumA=1 \QRnumE=\QRNswitchCost
 \fi
\fi
\QRnumB=4 \advance\QRnumE by -\QRnumB
\NUMBIN %type included
\QRnumA=\QRnumD\QRnumB=\QRnumE\NUMBIN %type block len included
\QRnumE=0 %cache len for ALPHANUM/NUMERIC
\GFReady
%\message{modelen:(\the\QRnumC)\binary}%
\let\PackQRTypeBlockNext=\PackQRTypeBlockLoop\PackQRTypeBlockNext}%\PACKcontent is issued at the block end
\def\PackQRTypeBlockLoop#1{%
\ifx\QRtype\QRtypeB
 \BINNUM#1\QRnumB=8 \NUMBIN
\else
 \QRnumG=\QRnumA\advance\QRnumE by 1
 \ALPHANUM#1\relax
 \ifx\QRtype\QRtypeA
  \multiply\QRnumG by 45 \advance\QRnumG by \QRnumA
  \QRnumA=\QRnumG
  \ifnum\QRnumE=2
   \QRnumB=11 \NUMBIN\QRnumE=\QRnumA %NUMBIN resets QRnumA
  \fi
 \else
  \multiply\QRnumG by 10 \advance\QRnumG by \QRnumA
  \QRnumA=\QRnumG
  \ifnum\QRnumE=3
   \QRnumB=10 \NUMBIN\QRnumE=\QRnumA %NUMBIN resets QRnumA
  \fi 
 \fi
\fi
\advance\QRnumD by -1
\ifnum\QRnumD=0 \let\PackQRTypeBlockNext=\PACKQRTypeBlockFlush\fi
\PackQRTypeBlockNext
}%
\def\PACKQRTypeBlockFlush{%
\ifnum\QRnumE>0
 \ifx\QRtype\QRtypeA
  \QRnumB=6 \NUMBIN
 \else
  \ifnum\QRnumE=1
   \QRnumB=4 \NUMBIN
  \else
   \QRnumB=7 \NUMBIN
  \fi
 \fi
\fi
\PACKcontent}%

\def\setQRversion#1{%\message{setQRversion \number#1}%
\edef\QRversion{<\romannumeral#1>}\!\let\!\QRside\!=\csname QRside\QRversion\endcsname\QRnumG=\QRside\advance\QRnumG by -1
\edef\QRmaxCoord{\the\QRnumG\QRspace}}%
\def\chooseMaxECLevel{\QRnumF=\csname QRBitLimit\QRversion Q\endcsname
\ifnum\QRnumC>\QRnumF
 \QRnumF=\csname QRBitLimit\QRversion M\endcsname
 \ifnum\QRnumC>\QRnumF
  \def\EClevel{L}%  
 \else
  \def\EClevel{M}%
 \fi
\else
 \QRnumF=\csname QRBitLimit\QRversion H\endcsname
 \ifnum\QRnumC>\QRnumF
  \def\EClevel{Q}%
 \else
  \def\EClevel{H}%
 \fi
\fi
}%
\let\chooseECLevel=\chooseMaxECLevel
\newif\ifVersionChooseLoopCont
\def\versionChooseLoopBreakTest{\ifnum\QRnumE<\QRmaxVersion\relax
 \ifnum\QRnumC>\QRnumF
  \VersionChooseLoopConttrue
 \else
  \VersionChooseLoopContfalse
 \fi
\else\VersionChooseLoopContfalse\fi}%
\def\chooseVersionToFit{\QRnumE=\QRminVersion\relax
 \loop \QRnumF=\csname QRBitLimit<\romannumeral\QRnumE>\endcsname
  \versionChooseLoopBreakTest
  \ifVersionChooseLoopCont
   \advance\QRnumE by 1 
 \repeat
}%
\def\fixVersion#1{\def\chooseVersion{\QRnumE=#1\relax}}%
\def\freeVersion{\let\chooseVersion=\chooseVersionToFit}%
\freeVersion
\def\PAD{%choses appropriate version and EC level, PADs data bits to full data size for the version and EC level
 \chooseVersion% choses \QRnumE
 \setQRversion{\QRnumE}%
 \QRnumF=\csname QRBitLimit\QRversion\endcsname
 \ifnum\QRnumC>\QRnumF
  \errmessage{No room for data in QRcode versions till \the\QRnumE}%
 \fi
 \chooseECLevel
 \!\let\!\QRBitLimit\!=\csname QRBitLimit\QRversion\EClevel\endcsname
 %\message{version \QRversion=\the\QRnumE, EClevel \EClevel, QRBitLimit \QRBitLimit}%
 \QRnumA=0 \QRnumB=\QRBitLimit\advance\QRnumB by -\QRnumC
 \ifnum\QRnumB<4 \else\QRnumB=4\NUMBIN\fi %PAD mode mark
 \QRnumB=\QRnumC\divide\QRnumB by 8 \multiply\QRnumB by 8
 \advance\QRnumB by-\QRnumC
 \ifnum\QRnumB=0 \else
  \advance\QRnumB by 8 \NUMBIN %PAD by 0 to the byte boundary
 \fi
 \let\PADnext=\PADbyXVI\PADnext %PADDING by alternating 2 PAD bytes
}%
\def\PADbyXVI{\QRnumB=\QRBitLimit\advance\QRnumB by -\QRnumC
\ifnum\QRnumB<16 \let\PADnext\PADlastVIII\else\QRnumB=16 \QRnumA="EC11 \NUMBIN\fi\PADnext}%
\def\PADlastVIII{\ifnum\QRnumB=8 \QRnumA="EC \NUMBIN\fi}%

%%%%% ReedSolomon LEVEL %%%%%
% split to blocks, find ec for the blocks, join blocks, join ec blocks 
\def\ReedSolomon{\QRSplitBlocks\QRMergeBlocks\QRMergeECs
%\message{After merge (\the\QRnumC)\binary QRversion = \QRversion}%
\QRnumB=\csname QRPadBits\QRversion\endcsname\QRnumA=0 
\NUMBIN
%\message{After PADDBits (\the\QRnumC)\binary QRversion = \QRversion}%
}%
% QRnumC does not contain the length of binary now.

\def\binarySplit#1#2#3#4#5#6#7#8{\!\def\!\binSplit\!{\binSplit#1#2#3#4#5#6#7#8}\advance\QRnumB by -1
\ifnum\QRnumB>0 \!\binarySplit\else\!\defbinary\fi}%
\def\defbinary#1\endbinary{\def\binary{#1}}%
%
\def\QRSplitBlocks{% The blocks would be numbered downwards. Block len differ by at most 1, the longer blocks are at the end.
\!\let\!\QRBlocks\!=\csname QRBlocks\QRversion\EClevel\endcsname
\!\let\!\QREC\!=\csname QREC\QRversion\EClevel\endcsname
\QRnumD=\QRBlocks\QRnumE=\QRnumC\divide \QRnumE by 8 %number of data bytes
\loop
\ifnum\QRnumD>1
 %\message{QRSplitBlocks remaining blocks \the\QRnumD, remaining bytes \the\QRnumE binary \binary}%
 \QRnumB=\QRnumE
 \divide\QRnumB by \QRnumD\advance\QRnumE by -\QRnumB\QRnumF=\QRnumB
 \def\binSplit{}\!\binarySplit\binary\endbinary
 \processSplit
 \advance\QRnumD by -1
\repeat
\let\binSplit=\binary
\QRnumF=\QRnumE
\processSplit
\def\binary{}%
}%

\def\processSplit{\!\let\csname bin\the\QRnumD\endcsname=\binSplit
%%%% Reed Solomon here %%%%
% ... not to compute the part lengths again, as GFP is represented by a set of definitions, it's better to finish the computation now.
% the arithmetic definition follows 
\QRnumA=\QREC
%\message{processSplit[\the\QRnumD] (\the\QRnumF)\binSplit}%
\binaryToGFP\GFPdata\binSplit\QRnumF
%\GFPshow\GFPdata\par
\GFPshiftLeft\QRnumA\GFPdata
%DataShifted \GFPshow\GFPdata\par
%\message{split ... QRnumA=\the\QRnumA}%
%Prod \!\GFPshow\csname GFPProd\the\QRnumA\endcsname\par
\!\GFPmod\!\GFPec\!\GFPdata\csname GFPProd\the\QRnumA\endcsname
\advance\QRnumA by -1\relax\edef\GFPecDeg{\the\QRnumA}% don't remove leading zeros!
%Data \GFPshow\GFPdata\par
%EC \GFPshow\GFPec\par
\ReedSolomonCheck
\!\GFPTobinary\csname EC\the\QRnumD\endcsname\GFPec
%\message{EC\the\QRnumD = \csname EC\the\QRnumD\endcsname}%
}%

\def\ReedSolomonCheck{%
{\!\xdef\csname GFPcheckDeg\endcsname{\csname GFPdataDeg\endcsname}%
\QRnumG=\csname GFPcheckDeg\endcsname
\loop
%\message{GFPcheck \the\QRnumG}%
\!\edef\csname GFPcheck[\the\QRnumG]\endcsname{\csname GFPdata[\the\QRnumG]\endcsname}\advance\QRnumG by -1 \ifnum\QRnumG>\QRnumA\repeat
\loop
%\message{GFPcheck(EC) \the\QRnumG}%
\!\edef\csname GFPcheck[\the\QRnumG]\endcsname{\csname GFPec[\the\QRnumG]\endcsname}\advance\QRnumG by -1 \ifnum\QRnumG>-1 \repeat
\QRnumG=\QRnumA\advance \QRnumG by 1
%Check \GFPshow\GFPcheck\par
\!\GFPmod\!\GFPtst\!\GFPcheck\csname GFPProd\the\QRnumG\endcsname
%Tst \GFPshow\GFPtst\par
}}%

\def\takeByte#1{\def\taken{#1}\ifx#1\QRempty\def\byte{}\else\!\takeEight#1\endtake\fi}
\def\takeEight#1#2#3#4#5#6#7#8#9\endtake{\def\byte{#1#2#3#4#5#6#7#8}\!\def\taken{#9}}%
\def\QRMergeBlocks{% Blocks should be merged (in our numbering taking one byte from each block (going downwards) until the shorter blocks are empty
\QRnumD=\QRBlocks
\loop
 \!\takeByte\csname bin\the\QRnumD\endcsname
 \!\!\!\def\!\!\!\binary\!\!\!{\!\binary\byte}%
\ifnum\QRnumD>1
 \advance\QRnumD by -1
\repeat
\!\ifx\csname bin1\endcsname\QRempty\else\!\QRMergeBlocks\fi
}%
\def\QRMergeECs{% ECs should be merged (in our numbering taking one byte from each block (going downwards) until they are empty
\QRnumD=\QRBlocks
\loop
 \!\takeByte\csname EC\the\QRnumD\endcsname
 \!\!\!\def\!\!\!\binary\!\!\!{\!\binary\byte}%
\ifnum\QRnumD>1
 \advance\QRnumD by -1
\repeat
\!\ifx\csname EC1\endcsname\QRempty\else\!\QRMergeECs\fi
}%Merge EC's should be added as well

%%%%% GF(0x11D) ARITHMETICS %%%%%
% fortunately we could work with 9 bit elements (even when multiplying by the generator)
% results in \GFacc
\def\GFmodulo{\1\0\0\0\1\1\1\0\1}%
\def\GFid{}%
\def\GFnegNine#1#2#3#4#5#6#7#8#9{\ifx#9\0\!#1\!#2\!#3\!#4\!#5\!#6\!#7\!#8\!\1\else\!#1\!#2\!#3\!#4\!#5\!#6\!#7\!#8\!\0\fi}
\def\GFxorZeroReady{\noexpand\GFid}%
\def\GFxorOneReady{\noexpand\GFnegNine}%
\def\GFxorReady{\let\0=\GFxorZeroReady\let\1=\GFxorOneReady}
\def\GFZeroReady{\noexpand\0}%
\def\GFOneReady{\noexpand\1}%
\def\GFReady{\let\0=\GFZeroReady\let\1=\GFOneReady}
\def\GFZeroGobble{}%
\def\GFOneGobble{\relax}% has to be different from GFZeroGobble
\def\GFGobble{\let\0=\GFZeroGobble\let\1=\GFOneGobble}
\def\GFadd#1#2#3#4#5#6#7#8#9{\GFxorReady\edef\GFacc{\GFacc}\GFReady
\edef\GFacc{\GFacc#1#2#3#4#5#6#7#8#9}}
\def\GFmod{\GFGobble\!\ifx\!\0\GFacc\else\!\GFadd\GFmodulo\fi}%
\def\GFx{\!\!\!\def\!\!\!\GFacc\!\!\!{\!\gobbleone\GFacc\0}\GFmod}%
\def\GFone{\0\0\0\0\0\0\0\0\1}
\def\GFzero{\0\0\0\0\0\0\0\0\0}
\def\GFLOGEXPloop{\!\!\!\def\!\csname\!G\!F\!E\!X\!P\!\the\!\QRnumA\!\endcsname\!{\GFacc}%
\def\0{0}\def\1{1}\!\edef\csname GFLOG\GFacc\endcsname{\the\QRnumA}%
\GFx\advance\QRnumA by 1 \ifnum\QRnumA=255 \let\GFLOGEXPnext=\relax\fi\GFLOGEXPnext}%
\let\GFLOGEXPnext=\GFLOGEXPloop\QRnumA=0\!\def\!\GFacc\!{\GFone}\GFLOGEXPnext
\def\GFlog{\def\0{0}\def\1{1}\QRnumA=\csname GFLOG\GFacc\endcsname\relax}%
\def\GFexp{\!\!\!\def\!\!\!\GFacc\!\!\!{\csname GFEXP\the\QRnumA\endcsname}}%
\def\GFmul#1#2#3#4#5#6#7#8#9{\ifx\GFacc\GFzero\else\GFlog
 \def\GFacc{#1#2#3#4#5#6#7#8#9}\GFmod\ifx\GFacc\GFzero\else
 \QRnumB=\QRnumA\GFlog\advance\QRnumA by \QRnumB
 \ifnum\QRnumA>254 \advance\QRnumA by -255 \fi
 \GFexp\fi\fi}%
\def\GFoneover{\GFlog\QRnumB=\QRnumA\ifnum\QRnumA=0 \else \QRnumA=255\advance\QRnumA by -\QRnumB\fi\GFexp}%
\def\GFshow{\def\0{0}\def\1{1}\GFacc}%
%
%%%% GFP ARITHMETIC %%%%
\def\GFPreset#1{\QRnumA=\csname\string#1Deg\endcsname\relax
\def\GFPresetnext{\!\!\!\gdef\!\csname\!\string\!#1\![\!\the\!\QRnumA\!]\!\endcsname\!{\GFzero}%
\advance\QRnumA by -1\ifnum\QRnumA<0\let\GFPresetnext=\relax\fi\GFPresetnext}\GFPresetnext}%
%
\def\GFPmul#1#2#3{\begingroup\escapechar=-1 \QRnumA=\csname\string#2Deg\endcsname\relax
\QRnumB=\csname\string#3Deg\endcsname\relax
\advance\QRnumA by \QRnumB\!\xdef\csname\string#1Deg\endcsname{\the\QRnumA}\GFPreset#1\relax
\def\GFPmulnextloop{\begingroup\QRnumC=\QRnumB\advance\QRnumC by\QRnumA\!\!\!\GFmul\csname\string#2[\the\QRnumA]\endcsname
\!\!\!\GFadd\csname\string#1[\the\QRnumC]\endcsname
\!\!\!\gdef\!\csname\!\string\!#1\![\!\the\!\QRnumC\!]\!\endcsname\!{\GFacc}%
\endgroup\advance\QRnumA by -1 \ifnum\QRnumA<0 \let\GFPmulnextnext=\relax\fi\GFPmulnextnext}%
\def\GFPmulnext{\!\!\!\def\!\!\!\GFacc\!\!\!{\csname\string#3[\the\QRnumB]\endcsname}\QRnumA=\csname\string#2Deg\endcsname
\let\GFPmulnextnext=\GFPmulnextloop\GFPmulnextnext
\advance\QRnumB by -1 \ifnum\QRnumB<0 \let\GFPmulnext=\relax\fi\GFPmulnext}\GFPmulnext\endgroup
}%
%
%GFPcopy expects #1 is not defined at all or was defined with smaller degree ... otherwise there remains non accessed allocation
\def\GFPcopy#1#2{\begingroup\escapechar=-1 \!\xdef\csname\string#1Deg\endcsname{\csname\string#2Deg\endcsname}%
\QRnumA=\csname\string#1Deg\endcsname
\def\GFPcopyloop{\!\!\!\!\!\!\!\gdef\!\!\!\csname\!\!\!\string\!\!\!#1\!\!\![\!\!\!\the\!\!\!\QRnumA\!\!\!]\!\!\!\endcsname\!\!\!{\csname\string#2[\the\QRnumA]\endcsname}\advance\QRnumA by -1 \ifnum\QRnumA<0 \else\!\GFPcopyloop\fi}%
\GFPcopyloop\endgroup}%
%
\def\GFPshiftLeft{\afterassignment\GFPshiftLeftCont\QRnumB=}%
\def\GFPshiftLeftCont#1{\begingroup\escapechar=-1
\QRnumA=\csname\string#1Deg\endcsname\relax
\advance\QRnumB by \QRnumA \!\xdef\csname\string#1Deg\endcsname{\the\QRnumB}%
\def\GFPshiftLeftLoop{\!\!\!\!\!\!\!\gdef\!\!\!\csname\!\!\!\string\!\!\!#1\!\!\![\!\!\!\the\!\!\!\QRnumB\!\!\!]\!\!\!\endcsname\!\!\!{\csname\string#1[\the\QRnumA]\endcsname}\advance\QRnumA by -1
\advance\QRnumB by -1 \ifnum\QRnumA<0 \else\!\GFPshiftLeftLoop\fi}%
\def\GFPshiftLeftClearLoop{\!\!\!\gdef\!\csname\!\string\!#1\![\!\the\!\QRnumB\!]\!\endcsname\!{\GFzero}\advance\QRnumB by -1 \ifnum\QRnumB<0 \else\!\GFPshiftLeftClearLoop\fi}%
\GFPshiftLeftLoop
\GFPshiftLeftClearLoop
\endgroup}%
%
\def\GFPgenloop{\GFPcopy\GFPPrevProd\GFPProd\!\def\csname GFPGen\the\QRnumA Deg\endcsname{1}%
\!\!\!\def\!\csname\!G\!F\!P\!G\!e\!n\!\the\!\QRnumA\![\!1\!]\!\endcsname\!{\GFone}%
\GFexp\!\!\!\def\!\csname\!G\!F\!P\!G\!e\!n\!\the\!\QRnumA\![\!0\!]\!\endcsname\!{\GFacc}%
\!\GFPmul\!\GFPProd\!\GFPPrevProd\csname GFPGen\the\QRnumA\endcsname
\advance\QRnumA by 1 \!\GFPcopy\csname GFPProd\the\QRnumA\endcsname\GFPProd
\ifnum\QRnumA<30 \!\GFPgenloop\fi}%
\def\GFPProdDeg{0}\!\!\!\def\!\csname\!G\!F\!P\!P\!r\!o\!d\![\!0\!]\!\endcsname\!{\GFone}%
\QRnumA=0 \!\GFPcopy\csname GFPProd\the\QRnumA\endcsname\GFPProd
\GFPgenloop
%
\def\GFPreducedeg#1{\begingroup\escapechar=-1
\def\GFPreducedegloop{\!\ifx\csname\string#1[\csname\string#1Deg\endcsname]\endcsname\GFzero
\QRnumA=\csname\string#1Deg\endcsname\relax\advance\QRnumA by -1
\ifnum\QRnumA<0 \else\!\xdef\csname\string#1Deg\endcsname{\the\QRnumA}\!\!\!\GFPreducedegloop\fi\fi}\GFPreducedegloop\endgroup}%
%
\def\GFPmod#1#2#3{\begingroup\escapechar=-1 \GFPcopy#1#2\relax
\GFPreducedeg#3\QRnumA=\csname\string#3Deg\endcsname\relax
\!\!\!\def\!\!\!\GFacc\!\!\!{\csname\string#3[\the\QRnumA]\endcsname}\GFoneover
\!\def\!\DivisorMonicFactor\!{\GFacc}% we would use monic polynomials anyway ... nonzero not tested!
\def\GFPmodloop{\GFPreducedeg#1\relax
\QRnumC=\csname\string#1Deg\endcsname\relax
\!\!\!\def\!\!\!\GFacc\!\!\!{\csname\string#1[\the\QRnumC]\endcsname}%
\advance\QRnumC by -\csname\string#3Deg\endcsname\relax % shift left
\ifnum\QRnumC<0 \else
\!\GFmul\DivisorMonicFactor
\def\GFPshiftsubloop{\begingroup\advance\QRnumC by \QRnumA
\!\!\!\GFmul\csname\string#3[\the\QRnumA]\endcsname
\!\!\!\GFadd\csname\string#1[\the\QRnumC]\endcsname
\!\!\!\gdef\!\csname\!\string\!#1\![\!\the\!\QRnumC\!]\!\endcsname\!{\GFacc}%
\endgroup
\advance\QRnumA by -1 \ifnum\QRnumA<0 \else\!\!\!\GFPshiftsubloop\fi
}\QRnumA=\csname\string#3Deg\endcsname\relax\GFPshiftsubloop
\!\GFPmodloop\fi}\GFPmodloop
\endgroup}%
%
\def\GFPshow#1{\begingroup\escapechar=-1\QRnumA=\csname\string#1Deg\endcsname
\def\GFPshowloop{\!\!\!\def\!\!\!\GFacc\!\!\!{\csname\string#1[\the\QRnumA]\endcsname}\GFshow\odot x^{\the\QRnumA}%
\advance\QRnumA by -1 \ifnum\QRnumA<0 \else\oplus\!\GFPshowloop\fi}$\GFPshowloop$\endgroup}%
%
\def\binaryToGFPloopend#1\endbinary{}%
\def\binaryToGFP#1#2#3{\begingroup\escapechar=-1 \QRnumA=#3\advance\QRnumA by -1 \!\xdef\csname\string#1Deg\endcsname{\the\QRnumA}%
\def\binaryToGFPloop##1##2##3##4##5##6##7##8{\!\gdef\csname\string#1[\the\QRnumA]\endcsname{\0##1##2##3##4##5##6##7##8}%
\advance\QRnumA by -1 \ifnum\QRnumA<0 \!\binaryToGFPloopend\else\!\binaryToGFPloop\fi}%
\!\binaryToGFPloop#2\endbinary
\endgroup}%
%
%\def\GFPTobinary#1{\iffalse{\fi\begingroup\def\0{\aftergroup\!\aftergroup\0}\def\1{\aftergroup\!\aftergroup\1}%
%\escapechar=-1
%\QRnumA=\csname\string#1Deg\endcsname\relax
%\aftergroup\!\aftergroup\!\aftergroup\!\aftergroup\def\aftergroup\!\aftergroup\!\aftergroup\!\aftergroup\binary\aftergroup\!\aftergroup\!\aftergroup\!\%aftergroup{\iffalse}\fi\aftergroup\!\aftergroup\binary
%\def\GFPTobinaryloop{\!\!\!\gobbleone\csname\string#1[\the\QRnumA]\endcsname\advance\QRnumA by -1 \ifnum\QRnumA<0 \else \!\GFPTobinaryloop\fi}%
%\GFPTobinaryloop
%\endgroup}}%
%
\def\GFPTobinary#1#2{%
\begingroup\escapechar=-1\def#1{}%
\QRnumA=\csname\string#2Deg\endcsname\relax
\def\GFPTobinaryloop{%
\!\let\!\nextkoef\!=\csname\string#2[\the\QRnumA]\endcsname
\!\edef\!#1\!{\!#1\!\gobbleone\nextkoef}%
\advance\QRnumA by -1 \ifnum\QRnumA<0 \else \!\GFPTobinaryloop\fi}%
\GFReady
\GFPTobinaryloop
\xdef\gbinary{\def\noexpand#1{#1}}% local assignment outside the group!
\endgroup
\gbinary}%
%
%%%%% GEOMETRY %%%%
%
%For each version there would be set of definitions ... one per pixel. 
%QRVER<\QRversion>[i,j] will be filled by one of: \QRBigFixed\0, \QRBigFixed\1, \QRFixed\0, \QRFixed\1, \QRecMaskInfo 0 ... \QRecMaskInfo E, \QRmasked<i><j>\0, \QRmasked<i><j>\1, ... always 2 tokens
% (QRBigFixed, QRFixed are static for given version, QRecMaskInfo depends on chosen encoddinglevel and mask)
% data pixels are initialised to \QRmasked<i><j>\1, could be changed using QRSet according to the data to be stored
%\QRSet sets the databits, \QRchooseMask choses the optimal mask and \QRSetBox fills the box with corresponding rules
% the idea to have QRBigFixed for Finders and (outer Aligns) is to print them separately by bigger rules. 
% It is not implemented yet and probably would never be. 
%There would be different modes how to use the information ... 
%%%% Modes of use %%%%
%\QRmasked is not redefined according the mode, Mask is changed instead
\def\QRShowBitsMode{\let\QRBigFixed=\relax\let\QRFixed=\relax\def\QRecMaskInfo##1{\csname QRecMask\QRMask\EClevel##1\endcsname}}
%%%% ^ for printing %%%%
\def\QRSetBitsMode{\QRSetMask8\let\QRBigFixed=\QRSetLoopTwo\let\QRFixed=\QRSetLoopTwo\let\QRecMaskInfo\QRSetLoopTwo}
%%%% ^ for QRSet ... the MASK8 would store the databits in QRmasked, everything else is skipped
%
%%%% QRSET %%%%
\newif\ifQRSecondColumn
%\binery should have exact len ... the number of \QRmasked items among QRVER
%\QRnumC is vertical direction of filling
\def\QRSetLoop{\ifQRSecondColumn\QRSecondColumnfalse\advance\QRnumA by 1 \advance\QRnumB by \QRnumC\else
\QRSecondColumntrue\advance\QRnumA by -1 \fi
\ifnum\QRnumB<0 \QRSetStartColumn\fi\ifnum\QRnumB=\QRside \QRSetStartColumn\fi
\ifnum\QRnumA<0 \else
\!\!\!\doubleToken\!\csname\!Q\!R\!V\!E\!R\!\QRversion\![\!\the\!\QRnumB\!,\!\the\!\QRnumA\!]\!\endcsname\fi}%
% expands to <token1><token2><QRVER<>[i,j]> where <QRVER<>[i,j]> expands to <token1><token2>
\def\QRSet{%\message{QRSet}%
\QRinit\QRSetBitsMode\QRSecondColumnfalse\QRnumC=-1 \QRnumA=\QRmaxCoord \QRnumB=\QRnumA
\!\!\!\doubleToken\!\csname\!Q\!R\!V\!E\!R\!\QRversion\![\!\the\!\QRnumB\!,\!\the\!\QRnumA\!]\!\endcsname\binary}%
% again expands to <token1><token2><QRVER<>[i,j]> where <QRVER<>[i,j]> expands to <token1><token2>
\def\QRSetLoopTwo#1#2{\QRSetLoop}% used for <token1> to skip <token2><QRVER<>[i,j]> (fixed patterns ignored while filling)
\def\QRSetBit#1#2#3{\!\!\!\def\!\!\!#2\!\!\!{\!\gobblesecond#2#3}\QRSetLoop}% Mask8 used for QRSet 
% QRSetBit<token2><QRVER<>[i,j]>#1 after first expandafter chain \!\def\!<QRVER<>[i,j]>\!{\gobblesecond<token1><token2>#1}\QRSetLoop
% after second expandafter chain \def<QRVER<>[i,j]>{<token1>#1}\QRSetLoop so 2nd token was replaced by first token from the replacement list
\def\QRSetStartColumn{\QRnumC=-\QRnumC\advance\QRnumB by \QRnumC\advance\QRnumA by -2 
\ifnum\QRnumA=6 \advance\QRnumA by -1 \fi}%
% revert direction, move left twice and skip column6, and return vertically once (as invoked outside the square)
%
%%%% ARRAY FILL %%%%
%%%% DEFINING THE VERSION SQUARE %%%%
\def\QRinit{%\message{QRinit version \QRversion \!\meaning\csname QRVER\QRversion[0,0]\endcsname}%
\GFGobble\!\ifx\!\relax\csname QRVER\QRversion[0,0]\endcsname\!\QRInit\fi}%
\def\QRInit{%\message{QRInit}%
\QRFill\QRPutFinders\QRPutTimers\QRPutAligns\QRPutVersion\QRPutECMaskInfo%\QRLog
}%
%
\def\QRFill{\QRnumA=\QRmaxCoord\QRnumC=\QRnumA\divide\QRnumC by 12 \multiply\QRnumC by -12 \advance\QRnumC by \QRnumA
\QRnumE=\QRnumC\ifnum\QRnumE>5 \advance\QRnumE by -6 \fi\QRFillLoop}% fill by QRmasked first, fixed patterns would replace it later
\def\QRFillLoop{\QRnumB=\QRmaxCoord\QRnumD=\QRnumE\QRFillLoopLoop
\advance\QRnumA by -1 \advance\QRnumC by -1
\ifnum\QRnumC<0 \advance\QRnumC by 12 \fi
\ifnum\QRnumA<0 \else\!\QRFillLoop\fi}%
\def\QRFillLoopLoop{\QRFillLoopLoopDo
\advance\QRnumB by -1 \advance\QRnumD by -1
\ifnum\QRnumD<0 \advance\QRnumD by 6 \fi
\ifnum\QRnumB<0 \else\!\QRFillLoopLoop\fi}%
\def\QRFillLoopLoopDo{\!\!\!\gdef\!\csname\!Q\!R\!V\!E\!R\!\QRversion\![\!\the\!\QRnumA\!,\!\the\!\QRnumB\!]\!\endcsname\!{\csname QRmasked\the\QRnumC,\the\QRnumD\endcsname\1}}%
%
\def\QRPutBasic#1#2#3#4{\!\gdef\csname QRVER\QRversion[\number#1,\number#2]\endcsname{#3#4}}%
%filling with main diagonal symmetry
\def\QRPutTwoBigFixed#1{\QRPutBasic\QRnumA\QRnumB\QRBigFixed#1\QRPutBasic\QRnumB\QRnumA\QRBigFixed#1}%
\def\QRPutBigFixedSym#1{\QRPutTwoBigFixed#1\relax
{\QRnumA=-\QRnumA\advance\QRnumA by \QRmaxCoord\QRPutTwoBigFixed#1}% with one mirror symmetry
{\QRnumB=-\QRnumB\advance\QRnumB by \QRmaxCoord\QRPutTwoBigFixed#1}% with the other mirror symmetry
}%
%
\def\QRPutBigFixedLine#1#2#3#4{\QRnumA=#1 \QRnumB=#2 \QRPutBigFixedLineLoop#3#4}%
\def\QRPutBigFixedLineLoop#1#2{\QRPutBigFixedSym#2\advance\QRnumB by -1 \ifnum\QRnumB<#1 \else\!\QRPutBigFixedLineLoop\!#1\!#2\fi}%
\def\QRPutFinders{\QRnumA=\QRside\advance\QRnumA by -1
\QRPutBigFixedLine770\0\relax
\QRPutBigFixedLine660\1\relax
\QRPutBigFixedLine050\1\relax
\QRPutBigFixedLine551\0\relax
\QRPutBigFixedLine141\0\relax
\QRPutBigFixedLine442\1\relax
\QRPutBigFixedLine332\1\relax
\QRPutBigFixedLine222\1\relax
}%
%
\def\QRPutFixed#1{\QRPutBasic\QRnumA\QRnumB\QRFixed#1}%
\def\QRPutTwoFixed#1{\QRPutFixed#1\QRPutBasic\QRnumB\QRnumA\QRFixed#1}%
%
\def\QRPutAlignOffs#1#2#3{{\advance\QRnumA by #1 \advance\QRnumB by #2 \QRPutBasic\QRnumA\QRnumB\QRBigFixed#3}}%
%
\def\QRPutAlignment#1#2{{\QRnumA=#1\relax\QRnumB=#2\relax\QRPutFixed\1%
\QRPutAlignOffs{-1}{-1}\0\QRPutAlignOffs{-1}0\0\QRPutAlignOffs{-1}1\0\QRPutAlignOffs01\0%
\QRPutAlignOffs11\0\QRPutAlignOffs10\0\QRPutAlignOffs1{-1}\0\QRPutAlignOffs0{-1}\0%
\QRPutAlignOffs{-2}{-2}\1\QRPutAlignOffs{-2}{-1}\1\QRPutAlignOffs{-2}0\1\QRPutAlignOffs{-2}1\1%
\QRPutAlignOffs{-2}2\1\QRPutAlignOffs{-1}2\1\QRPutAlignOffs02\1\QRPutAlignOffs12\1%
\QRPutAlignOffs22\1\QRPutAlignOffs21\1\QRPutAlignOffs20\1\QRPutAlignOffs2{-1}\1%
\QRPutAlignOffs2{-2}\1\QRPutAlignOffs1{-2}\1\QRPutAlignOffs0{-2}\1\QRPutAlignOffs{-1}{-2}\1%
}}%
\def\QRPutAligns{\QRnumD=\csname QRSideAligns\QRversion\endcsname
%\message{QRPutAligns alignside \the\QRnumD}%
\ifnum\QRnumD>1 %compute the alignments distance
\QRnumE=\QRside\advance\QRnumE by -13 
%\message{Side-13 \the\QRnumE}%
\multiply\QRnumE by 5 
%\message{5*(side-13) \the\QRnumE}%
\QRnumF=7 \multiply\QRnumF by \QRnumD
\advance\QRnumE by \QRnumF
%\message{7A \the\QRnumF; 5*(side-13)+7A \the\QRnumE}%
\divide\QRnumE by 10 \divide\QRnumE by \QRnumD
%\message{5*(side-13)/(10A)+7/10 \the\QRnumE}%
\multiply\QRnumE by 2 
%\message{Aligns dist \the\QRnumE}%
\fi%
\QRnumA=\QRside\advance\QRnumA by -7 \QRnumB=\QRnumA\QRnumC=\QRnumD
\ifnum\QRnumD>0 \QRPutAlignsLoop\fi}%
\def\QRPutAlignsLoop{\loop
%\message{PutAlignments loop \the\QRnumD \the\QRnumE}%
\QRPutAlignsLoopLoop\advance\QRnumD by -1\advance\QRnumB by -\QRnumE\ifnum\QRnumD>0
\QRPutAlignment6\QRnumB \QRPutAlignment\QRnumB6
\repeat}%
\def\QRPutAlignsLoopLoop{{\loop
%\message{PutAlignments inner loop \the\QRnumA,\the\QRnumC}%
\QRPutAlignment\QRnumA\QRnumB
\advance\QRnumC by -1 \advance\QRnumA by -\QRnumE\ifnum\QRnumC>0 
\repeat}}%
%
\def\QRPutVersion{{\GFReady\QRnumA=\QRside\advance\QRnumA by -17 \divide\QRnumA by 4 
\ifnum\QRnumA>6 %no version info for versions upto 6
\binaryReset\QRnumB=6 \NUMBIN\let\binSplit=\binary
\QRnumA=0 %already was
\QRnumB=12 \NUMBIN %multiplied ... get ready for modloop
\!\QRVersionBCHmodLoop\binary\endbinary % reminder computed
\edef\binary{\binSplit\binary}%
\QRnumB=\QRside\advance\QRnumB by -9 \QRnumA=5
\!\QRVersionPutBitsLoop\binary % puts the bits to appropriate places!
\fi}}%
\def\QRVersionBCHmodLoop#1#2\endbinary{\ifx#1\0\def\binary{#2}\else
\edef\binary{\XorOOOOZZOZZOZO #2\endbinary}\fi
\advance\QRnumC by -1 \ifnum\QRnumC>12 \!\!\!\QRVersionBCHmodLoop\!\binary\!\endbinary\fi}%
%0x1f25=1111100100101 is the version modulo
\def\XorOOOOZZOZZOZO#1#2#3#4#5#6#7#8#9\endbinary{\QRNeg#1\QRNeg#2\QRNeg#3\QRNeg#4#5#6\QRNeg#7#8\XorZOZO#9\endbinary}%
\def\XorZOZO#1#2#3#4#5\endbinary{#1\QRNeg#2#3\QRNeg#4#5}%
%
\def\QRVersionPutBitsLoop#1#2#3{{%\message{QRVersionPutBits \the\QRnumA}%
\QRPutTwoFixed#1\advance\QRnumB by -1 \QRPutTwoFixed#2\advance\QRnumB by -1 \QRPutTwoFixed#3}%
\advance\QRnumA by -1 \ifnum\QRnumA<0 \else\!\QRVersionPutBitsLoop\fi}%
%
\def\QRPutTimers{\QRnumA=-8 \advance\QRnumA by\QRmaxCoord\QRnumB=6 \QRPutTimersLoop
\QRnumA=-7 \advance\QRnumA by \QRmaxCoord\QRnumB=8 \QRPutFixed\1}%
\def\QRPutTimersLoop{%
\QRPutTwoFixed\1\advance\QRnumA by -1
\ifnum\QRnumA<8 \else
\QRPutTwoFixed\0\advance\QRnumA by -1
\!\QRPutTimersLoop\fi}%

\def\QRPutECMaskInfoBit#1{\QRPutBasic\QRnumA\QRnumB\QRecMaskInfo#1}%
\def\QRPutECMaskInfoBitPos#1#2#3{\QRPutBasic#1#2\QRecMaskInfo#3}%
\def\QRPutECMaskInfo{\QRnumB=\QRmaxCoord\QRnumA=8
\QRPutECMaskInfoBit0\QRPutECMaskInfoBitPos080\relax
\advance\QRnumB by -1 \QRPutECMaskInfoBit1\QRPutECMaskInfoBitPos181\relax
\advance\QRnumB by -1 \QRPutECMaskInfoBit2\QRPutECMaskInfoBitPos282\relax
\advance\QRnumB by -1 \QRPutECMaskInfoBit3\QRPutECMaskInfoBitPos383\relax
\advance\QRnumB by -1 \QRPutECMaskInfoBit4\QRPutECMaskInfoBitPos484\relax
\advance\QRnumB by -1 \QRPutECMaskInfoBit5\QRPutECMaskInfoBitPos585\relax
\advance\QRnumB by -1 \QRPutECMaskInfoBit6\QRPutECMaskInfoBitPos786\relax
\advance\QRnumB by -1 \QRPutECMaskInfoBit7\QRPutECMaskInfoBitPos887\relax
\QRnumA=\QRnumB\QRnumB=8\relax
\advance\QRnumA by 1 \QRPutECMaskInfoBit8\QRPutECMaskInfoBitPos878\relax
\advance\QRnumA by 1 \QRPutECMaskInfoBit9\QRPutECMaskInfoBitPos859\relax
\advance\QRnumA by 1 \QRPutECMaskInfoBit A\QRPutECMaskInfoBitPos84A\relax
\advance\QRnumA by 1 \QRPutECMaskInfoBit B\QRPutECMaskInfoBitPos83B\relax
\advance\QRnumA by 1 \QRPutECMaskInfoBit C\QRPutECMaskInfoBitPos82C\relax
\advance\QRnumA by 1 \QRPutECMaskInfoBit D\QRPutECMaskInfoBitPos81D\relax
\advance\QRnumA by 1 \QRPutECMaskInfoBit E\QRPutECMaskInfoBitPos80E\relax
}%
%
%%%% DEFINING THE MASKS %%%%
\def\QRId{}%
\def\QRNeg#1{\ifx#1\0\!\1\else\!\0\fi}%
\def\QRSetMask#1{\edef\QRMask{<#1>}}% masks 0 ... 11 x 0 ... 5 should be set for #1
%%%% MASK PATTERNS %%%%
\def\QRdefMasks{%\message{QRdefMasks}%
\QRnumA=5 \QRdefMasksLoop}%
\def\QRdefMasksLoop{\QRnumB=11 \QRdefMasksLoopLoop\advance\QRnumA by -1 \ifnum\QRnumA<0 \else\!\QRdefMasksLoop\fi}%
\def\QRdefMasksLoopLoop{\QRdefMasksLoopLoopDo\advance\QRnumB by -1 \ifnum\QRnumB<0 \else\!\QRdefMasksLoopLoop\fi}%
\def\QRdefMasksLoopLoopDo{\!\edef\csname QRmasked\the\QRnumB,\the\QRnumA\endcsname{\noexpand\csname  QRmasked\noexpand\QRMask\the\QRnumB,\the\QRnumA\noexpand\endcsname}%
\QRnumC=\QRnumB\advance\QRnumC by \QRnumA
\ifodd\QRnumC
 \!\let\csname QRmasked<0>\the\QRnumB,\the\QRnumA\endcsname=\QRId
\else
 \!\let\csname QRmasked<0>\the\QRnumB,\the\QRnumA\endcsname=\QRNeg
\fi
\ifodd\QRnumB
 \!\let\csname QRmasked<1>\the\QRnumB,\the\QRnumA\endcsname=\QRId
\else
 \!\let\csname QRmasked<1>\the\QRnumB,\the\QRnumA\endcsname=\QRNeg
\fi
\QRnumD=\QRnumA
\divide\QRnumD by 3 \multiply\QRnumD by 3
\ifnum\QRnumD=\QRnumA
 \!\let\csname QRmasked<2>\the\QRnumB,\the\QRnumA\endcsname=\QRNeg
\else
 \!\let\csname QRmasked<2>\the\QRnumB,\the\QRnumA\endcsname=\QRId
\fi
\QRnumD=\QRnumC
\divide\QRnumD by 3 \multiply\QRnumD by 3
\ifnum\QRnumD=\QRnumC
 \!\let\csname QRmasked<3>\the\QRnumB,\the\QRnumA\endcsname=\QRNeg
\else
 \!\let\csname QRmasked<3>\the\QRnumB,\the\QRnumA\endcsname=\QRId
\fi
\QRnumD=\QRnumA\divide\QRnumD by 3 \multiply\QRnumD by 2
\advance\QRnumD by \QRnumB
\divide\QRnumD by 2
\ifodd\QRnumD
 \!\let\csname QRmasked<4>\the\QRnumB,\the\QRnumA\endcsname=\QRId
\else
 \!\let\csname QRmasked<4>\the\QRnumB,\the\QRnumA\endcsname=\QRNeg
\fi
\QRnumD=\QRnumA\multiply\QRnumD by \QRnumB
\QRnumE=\QRnumD
\divide\QRnumE by 3 \multiply\QRnumE by -3
\advance\QRnumE by \QRnumD
\advance\QRnumC by \QRnumE
\ifodd\QRnumD \advance\QRnumE by 1 \fi
\ifnum\QRnumE=0 
 \!\let\csname QRmasked<5>\the\QRnumB,\the\QRnumA\endcsname=\QRNeg
\else
 \!\let\csname QRmasked<5>\the\QRnumB,\the\QRnumA\endcsname=\QRId
\fi
\ifodd\QRnumE
 \!\let\csname QRmasked<6>\the\QRnumB,\the\QRnumA\endcsname=\QRId
\else
 \!\let\csname QRmasked<6>\the\QRnumB,\the\QRnumA\endcsname=\QRNeg
\fi
\ifodd\QRnumC
 \!\let\csname QRmasked<7>\the\QRnumB,\the\QRnumA\endcsname=\QRId
\else
 \!\let\csname QRmasked<7>\the\QRnumB,\the\QRnumA\endcsname=\QRNeg
\fi
\!\let\csname QRmasked<8>\the\QRnumB,\the\QRnumA\endcsname=\QRSetBit
\!\let\csname QRmasked<9>\the\QRnumB,\the\QRnumA\endcsname=\QRId
}%
\QRdefMasks
%
%%%% Precomputing ECMask encodding %%%%
%for all possible EC levels and masks ... for example \csname QRecMask<4>Q3\endcsname denotes value of 3rd bit for mask 4 EC level Q.
% EC level and MASK info is stored masked with pattern 101010000010010.
\def\QRSetecMaskOnMlK#1#2#3{\!\edef\csname QRecMask<#1>#2E\endcsname{\QRNeg#3}\QRSetecMasknMlK#1#2}%
\def\QRSetecMasknMlK#1#2#3{\!\def\csname QRecMask<#1>#2D\endcsname{#3}\QRSetecMaskMlK#1#2}%
\def\QRSetecMaskMlK#1#2#3{\!\edef\csname QRecMask<#1>#2C\endcsname{\QRNeg#3}\QRSetecMasklK#1#2}%
\def\QRSetecMasklK#1#2#3{\!\def\csname QRecMask<#1>#2B\endcsname{#3}\QRSetecMaskK#1#2}%
\def\QRSetecMaskK#1#2#3{\!\edef\csname QRecMask<#1>#2A\endcsname{\QRNeg#3}}%
\def\QRSetecMaskjihgfEdcBa#1#2#3{\!\def\csname QRecMask<#1>#29\endcsname{#3}\QRSetecMaskihgfEdcBa#1#2}%
\def\QRSetecMaskihgfEdcBa#1#2#3{\!\def\csname QRecMask<#1>#28\endcsname{#3}\QRSetecMaskhgfEdcBa#1#2}%
\def\QRSetecMaskhgfEdcBa#1#2#3{\!\def\csname QRecMask<#1>#27\endcsname{#3}\QRSetecMaskgfEdcBa#1#2}%
\def\QRSetecMaskgfEdcBa#1#2#3{\!\def\csname QRecMask<#1>#26\endcsname{#3}\QRSetecMaskfEdcBa#1#2}%
\def\QRSetecMaskfEdcBa#1#2#3{\!\def\csname QRecMask<#1>#25\endcsname{#3}\QRSetecMaskEdcBa#1#2}%
\def\QRSetecMaskEdcBa#1#2#3{\!\edef\csname QRecMask<#1>#24\endcsname{\QRNeg#3}\QRSetecMaskdcBa#1#2}%
\def\QRSetecMaskdcBa#1#2#3{\!\def\csname QRecMask<#1>#23\endcsname{#3}\QRSetecMaskcBa#1#2}%
\def\QRSetecMaskcBa#1#2#3{\!\def\csname QRecMask<#1>#22\endcsname{#3}\QRSetecMaskBa#1#2}%
\def\QRSetecMaskBa#1#2#3{\!\edef\csname QRecMask<#1>#21\endcsname{\QRNeg#3}\QRSetecMaska#1#2}%
\def\QRSetecMaska#1#2#3{\!\def\csname QRecMask<#1>#20\endcsname{#3}}%
\def\QRSetecMask#1#2{%mask eclevel
\binaryReset\QRnumA=#1 %MASKid as first parameter
\if#2L\advance\QRnumA by 8 \fi
\if#2H\advance\QRnumA by 16 \fi
\if#2Q\advance\QRnumA by 24 \fi
% M->0,L->1,H->2,Q->3, EC<<3|MASK
\QRnumB=5
\NUMBIN
%now \binary contains what needs to be BCH with "537=101 0011 0111 and masked with "5412 let us start with setting the higher bits ... masked by 101 01
\GFReady\!\QRSetecMaskOnMlK\!#1\!#2\binary
%so only last 10 digits of the reminder should be added ... masked by 00 0001 0010
\QRnumA=0 %already was
\QRnumB=10 \NUMBIN %multiplied ... get ready for modloop
\!\QRSetecMaskBCHmodLoop\binary\endbinary % reminder computed
\!\QRSetecMaskjihgfEdcBa\!#1\!#2\binary
%last 10 digits of the reminder added ... masked by 00 0001 0010
}%
\def\QRSetecMaskBCHmodLoop#1#2\endbinary{%
\ifx#1\0\def\binary{#2}\else
\edef\binary{\XorZOZZOOZOOO #2\endbinary}\fi
\advance\QRnumC by -1 \ifnum\QRnumC>10 \!\!\!\QRSetecMaskBCHmodLoop\!\binary\!\endbinary\fi}%
%0x537=10100110111 is the mask modulo
\def\XorZOZZOOZOOO#1#2#3#4#5#6#7#8#9\endbinary{#1\QRNeg#2#3#4\QRNeg#5\QRNeg#6#7\QRNeg#8\XorOO#9\endbinary}%
\def\XorOO#1#2#3\endbinary{\QRNeg#1\QRNeg#2#3}
\def\QRSetecMaskEC#1{\QRSetecMask0#1\QRSetecMask1#1\QRSetecMask2#1\QRSetecMask3#1\QRSetecMask4#1\QRSetecMask5#1\QRSetecMask6#1\QRSetecMask7#1}%
\QRSetecMaskEC L\QRSetecMaskEC M\QRSetecMaskEC Q\QRSetecMaskEC H
%
%%%% DEMERITS COMPUTATION FOR GIVEN MASK %%%%
\def\countQRDemerits{\QRShowBitsMode\QRnumB=0 \QRnumC=\QRnumB\QRnumD=\QRnumB
\countRowQRDemeritsAndDark\countColQRDemerits\countDarkQRDemerits\countSquareQRDemerits}% result in \QRnumC
%
\def\QRcountStaticZero#1{\ifx#1\0\advance\QRnumB by 1 \else
 \ifnum\QRnumB>3 \advance\QRnumB by -1 \advance\QRnumC by \QRnumB\fi
 \QRnumB=0
\fi#1}%
\def\QRcountStaticOne#1{\ifx#1\1\advance\QRnumB by 1 \else
 \ifnum\QRnumB>3 \advance\QRnumB by -1 \advance\QRnumC by \QRnumB\fi
 \QRnumB=0
\fi#1}%
\def\QRcountStatic{\let\0=\QRcountStaticZero\let\1=\QRcountStaticOne}%

\def\QRcountDarkZero{\advance\QRnumD by -1 }%
\def\QRcountDarkOne{\advance\QRnumD by 1 }%
\def\QRcountDark{\let\0=\QRcountDarkZero\let\1=\QRcountDarkOne}%

\def\QRtoBinaryOne{\!\def\!\binary\!{\binary\1}}%
\def\QRtoBinaryZero{\!\def\!\binary\!{\binary\0}}%
\def\QRtoBinary{\let\0\QRtoBinaryZero\let\1\QRtoBinaryOne}%

\def\QRtstSquareZero{\0\0\0\0}%
\def\QRtstSquareOne{\1\1\1\1}%
%
\def\countBinaryStaticDemerits{\QRcountStatic\QRnumB=0 \binary\relax}%
\def\countBinaryPatternOneDemerits#1\0\0\0\0\1\0\1\1\1\0\1#2\relax{%
\def\tmp{#2}\ifx\tmp\QRempty\else\advance\QRnumC by 40 \!\!\!\countBinaryPatternOneDemerits\!\tmp\!\relax\fi}
\def\countBinaryPatternTwoDemerits#1\1\0\1\1\1\0\1\0\0\0\0#2\relax{%
\def\tmp{#2}\ifx\tmp\QRempty\else\advance\QRnumC by 40 \!\!\!\countBinaryPatternTwoDemerits\!\tmp\!\relax\fi}
\def\countBinaryPatternThreeDemerits#1\0\0\0\0\1\0\1\1\1\0\1\0\0\0\0#2\relax{%
\def\tmp{#2}\ifx\tmp\QRempty\else\advance\QRnumC by -40 \!\!\!\countBinaryPatternThreeDemerits\!\!\!\0\!\!\!\0\!\!\!\0\!\!\!\0\!\tmp\!\relax\fi}%
\def\countBinaryPatternDemerits#1\relax{\countBinaryPatternOneDemerits#1\2\0\0\0\0\1\0\1\1\1\0\1\relax
\countBinaryPatternTwoDemerits#1\2\1\0\1\1\1\0\1\0\0\0\0\relax
\countBinaryPatternThreeDemerits#1\2\0\0\0\0\1\0\1\1\1\0\1\0\0\0\0\relax}%
\def\countBinaryDemerits{\countBinaryStaticDemerits
%\the\QRnumA After static demerits: \the\QRnumC\par
\!\countBinaryPatternDemerits\binary\relax
%\the\QRnumA After pattern demerits: \the\QRnumC\par
}%
\def\countRowQRDemeritsAndDark{%Rows:
\QRnumA=\QRmaxCoord\countRowQRDemeritsAndDarkLoop}%
\def\countRowQRDemeritsAndDarkLoop{\QRRowToBinary%\binaryShow
\countBinaryDemerits\QRcountDark\binary\advance\QRnumA by -1 \ifnum\QRnumA<0 \else\!\countRowQRDemeritsAndDarkLoop\fi}%
\def\countColQRDemerits{%Cols:
\QRnumA=\QRmaxCoord\countColQRDemeritsLoop}%
\def\countColQRDemeritsLoop{\QRColToBinary%\binaryShow
\countBinaryDemerits\advance\QRnumA by -1 \ifnum\QRnumA<0 \else\!\countColQRDemeritsLoop\fi}%
\def\countDarkQRDemerits{\ifnum\QRnumD<0 \QRnumD=-\QRnumD\fi\multiply\QRnumD by 10
\QRnumB=\QRside\multiply\QRnumB by \QRside
\divide\QRnumD by \QRnumB\multiply\QRnumD by 10
\advance\QRnumC by \QRnumD
}%After dark demerits: \the\QRnumC\par}%
\def\QRRowToBinary{\QRtoBinary\def\binary{}\QRnumB=\QRmaxCoord\QRRowToBinaryLoop}%
\def\QRRowToBinaryLoop{\csname QRVER\QRversion[\the\QRnumA,\the\QRnumB]\endcsname\advance\QRnumB by -1
\ifnum\QRnumB<0 \else\!\QRRowToBinaryLoop\fi}%
\def\QRColToBinary{\QRtoBinary\def\binary{}\QRnumB=\QRmaxCoord\QRColToBinaryLoop}%
\def\QRColToBinaryLoop{\csname QRVER\QRversion[\the\QRnumB,\the\QRnumA]\endcsname\advance\QRnumB by -1
\ifnum\QRnumB<0 \else\!\QRColToBinaryLoop\fi}%
\def\countSquareQRDemerits{\QRtoBinary\QRnumA=\QRmaxCoord\countSquareQRDemeritsLoop}% After squares: \the\QRnumC\par}%
\def\countSquareQRDemeritsLoop{\QRnumB=\QRmaxCoord\countSquareQRDemeritsLoopLoop\advance\QRnumA by -1 \ifnum\QRnumA=0 \else\!\countSquareQRDemeritsLoop\fi}%
\def\countSquareQRDemeritsLoopLoop{\countSquareQRDemeritsLoopLoopDo\advance\QRnumB by -1 \ifnum\QRnumB=0 
\else\!\countSquareQRDemeritsLoopLoop\fi}
\def\countSquareQRDemeritsLoopLoopDo{\def\binary{}%
\csname QRVER\QRversion[\the\QRnumA,\the\QRnumB]\endcsname\advance\QRnumA by -1
\csname QRVER\QRversion[\the\QRnumA,\the\QRnumB]\endcsname\advance\QRnumB by -1
\csname QRVER\QRversion[\the\QRnumA,\the\QRnumB]\endcsname\advance\QRnumA by 1
\csname QRVER\QRversion[\the\QRnumA,\the\QRnumB]\endcsname\advance\QRnumB by 1
\ifx\binary\QRtstSquareZero\advance\QRnumC by 3 \fi
\ifx\binary\QRtstSquareOne\advance\QRnumC by 3 \fi
}%
%%%%% MASK select %%%%%
\def\QRSelectMask{\QRSetMask7\countQRDemerits\def\QRbestMask{7}\QRnumE=\QRnumC
\SelectQRMaskIfBetter6\SelectQRMaskIfBetter5\SelectQRMaskIfBetter4\SelectQRMaskIfBetter3
\SelectQRMaskIfBetter2\SelectQRMaskIfBetter1\SelectQRMaskIfBetter0\QRSetMask{\QRbestMask}}%
%
\def\SelectQRMaskIfBetter#1{%Best demerits: \QRnumE
\QRSetMask#1\countQRDemerits
%#1 demerits: \the\QRnumC\par
%\message{SelectQRMaskIfBetter prev \the\QRnumE, now \the\QRnumC}%
\ifnum\QRnumC<\QRnumE\def\QRbestMask{#1}\QRnumE=\QRnumC\fi}%

%%%% Boxing LEVEL %%%%
\newdimen\QRbounding\newdimen\QRinside\newdimen\QRpixelUnit
\newdimen\basePixelUnit\basePixelUnit=1in \divide\basePixelUnit by \DPI\relax
%
\def\QRdimen{\afterassignment\QRdimenCont\QRbounding}%
\def\QRdimenCont{\defQRversionSizes}
\def\defQRversionSizes{\QRnumA=40 \loop\!\let\csname defQRVER<\romannumeral\QRnumA>Size\endcsname=\defQRversionSize
\advance\QRnumA by -1 \ifnum\QRnumA>0 \repeat}
\def\defQRversionSize{\QRnumC=\QRside\advance\QRnumC by 8
\QRinside=\QRbounding\divide\QRinside by \QRnumC
\divide\QRinside by\basePixelUnit\multiply\QRinside by \basePixelUnit
\QRpixelUnit=\QRinside\!\edef\csname QRVER\QRversion PixelUnit\endcsname{\the\QRpixelUnit}%
\QRnumC=\QRside\multiply\QRinside by \QRnumC
%\message{defQRversionSize QRinside: \the\QRinside}%
\!\let\csname defQRVER<\romannumeral\QRnumA>Size\endcsname=\relax
\!\edef\csname QRVER\QRversion inside\endcsname{\the\QRinside}}%
%
\QRdimen=5cm % good practice is to use space or \relax to invoke afterassignment before next another macro is started
\def\QRSetBox{%\message{QRSetBox QRversion=\QRversion}%
\QRShowBitsMode\csname defQRVER\QRversion Size\endcsname
\QRinside=\csname QRVER\QRversion inside\endcsname
\QRpixelUnit=\csname QRVER\QRversion PixelUnit\endcsname
\QRnumA=\QRside\advance\QRnumA by 4
\setbox\QRcodebox=\vbox{\hrule height 0pt
  \hbox to \QRbounding{\vrule  width 0pt height \QRnumA\QRpixelUnit depth 4\QRpixelUnit
    \hss\hbox to \QRinside{\hss\QRShowContent\hss}\hss
    \vrule width 0pt }
    \hrule height 0pt
   }%
}%
\def\QRShowContent{{%\tracingall
\def\1{\rlap{\kern\QRnumA\QRpixelUnit\raise\QRnumC\QRpixelUnit\hbox{\vrule width \the\QRpixelUnit\space height \the\QRpixelUnit\space depth 0pt}}}%
\let\0=\relax
\QRnumA=\QRmaxCoord\QRShowContentLoop}}%
\def\QRShowContentLoop{\QRnumC=0 \QRnumB=\QRmaxCoord\QRShowContentLoopLoop\advance\QRnumA by -1 \ifnum\QRnumA<0 \else\!\QRShowContentLoop\fi}%
\def\QRShowContentLoopLoop{\QRShowContentLoopLoopDo\advance\QRnumB by -1 \advance\QRnumC by 1 \ifnum\QRnumB<0 \else\!\QRShowContentLoopLoop\fi}%
\def\QRShowContentLoopLoopDo{\csname QRVER\QRversion[\the\QRnumB,\the\QRnumA]\endcsname}%
\endinput
%%%%%%%%%%%%%%%%%% example formats %%%%%%%%%%%%%%%
\QRdata|MECARD:N:Jméno Pøíjmení Titul za, Titul pøed;ADR:Ulice èíslo popisné/orientaèní, PSÈ Mìsto;ORG:Firma, Ulice kde firma sídlí èíslo popisné/orientaèní, PSÈ Mìsto firmy;URL:adresa www stránek;TEL:+420telefon|\QRcode 
\box\QRcodebox
\def\QRPstart{SPD*1.0}%
\def\QRPacc{*ACC:CZ72bbbbppppppuuuuuuuuuu}%
\def\QRPdspl{*DT:yyyymmdd}% could be empty
\def\QRPss{*X-SS:ssssssssss}% could be empty
\def\QRPamt{*AM:k.hh*CC:CZK}% almost arbitrary length of k's
\QRdata|*MSG:any message be cerefull with spaces and other special chars so QRdata is better method|\!\def\!\QRPmsg\!{\QRcontent}%
\def\QRPvs{*X-VS:vvvvvvvvvv}% could be empty, but not recommended
\def\QRPurl{*X-URL:HTTP://server/directory}%
\def\QRPay{\edef\QRPcontent{|\QRPstart\QRPacc\QRPamt\QRPdspl\QRPmsg\QRPvs\QRPss\QRPurl|}%
\!\QRdata\QRPcontent\QRcode}%
\QRPay\box\QRcodebox
%%%%%%%%%%%%%%% tests (almost removed) %%%%%%%%%%%
\def\QRLog{\QRnumA=\QRmaxCoord\QRLogLoop}%
\def\QRLogLoop{\QRnumB=\QRmaxCoord\QRLogLoopLoop\advance\QRnumA by -1 \ifnum\QRnumA<0 \else\!\QRLogLoop\fi}%
\def\QRLogLoopLoop{\QRLogLoopLoopDo\advance\QRnumB by -1 \ifnum\QRnumB<0 \else\!\QRLogLoopLoop\fi}%
\def\QRLogLoopLoopDo{\!\show\csname QRVER\QRversion[\the\QRnumB,\the\QRnumA]\endcsname}%
%\bye
\QRversion\box\QRcodebox\par\vfil\eject
\QRSetMinECLevel{H}%
\fixVersion{8}
\def\showdo#1{\show#1#1}%
\QRdata|QR code example 3.14159265358979323846264338327950|\QRcode $\pi$\QRversion \box\QRcodebox\par 
\freeVersion\QRcode  $\pi$\QRversion \box\QRcodebox\par
\def\QRminVersion{7}%
\def\QRmaxVersion{9}%
\QRcode  $\pi$\QRversion \box\QRcodebox\par
\def\QRmaxVersion{40}%

\QRdataadd,abcdefghi2,%\QRcode \par \QRversion\box\QRcodebox\par\vfil\eject
\QRdataadd,abcdefghi3,%\QRcode \par \QRversion\box\QRcodebox\par\vfil\eject
\QRdataadd,abcdefghi4,%\QRcode \par \QRversion\box\QRcodebox\par\vfil\eject
\QRdataadd,abcdefghi5,%\QRcode \par \QRversion\box\QRcodebox\par\vfil\eject
\QRdataadd,abcdefghi6,%\QRcode \par \QRversion\box\QRcodebox\par\vfil\eject
\QRdataadd,abcdefghi7,%\QRcode \par \QRversion\box\QRcodebox\par\vfil\eject
\QRdataadd,abcdefghi8,%\QRcode \par \QRversion\box\QRcodebox\par\vfil\eject
\QRdataadd,abcdefghijklmnopqrs9,%\QRcode \par \QRversion\box\QRcodebox\par\vfil\eject
\QRdataadd,abcdefghijklmnopqrsA,%\QRcode \par \QRversion\box\QRcodebox\par\vfil\eject
\QRdataadd,abcdefghijklmnopqrsB,%\QRcode \par \QRversion\box\QRcodebox\par\vfil\eject
\QRdataadd,abcdefghijklmnopqrsC,%\QRcode \par \QRversion\box\QRcodebox\par\vfil\eject
\QRdataadd,abcdefghijklmnopqrsD,%\QRcode \par \QRversion\box\QRcodebox\par\vfil\eject
\QRdataadd,abcdefghijklmnopqrsE,%\QRcode \par \QRversion\box\QRcodebox\par\vfil\eject
\QRdataadd,abcdefghijklmnopqrstF,%\QRcode \par \QRversion\box\QRcodebox\par\vfil\eject
\QRdataadd,abcdefghijklmnopqrstuG,%\QRcode \par \QRversion\box\QRcodebox\par\vfil\eject
\QRdataadd,abcdefghijklmnopqrstuvwH,%\QRcode \par \QRversion\box\QRcodebox\par\vfil\eject
\QRdataadd,abcdefghijklmnopqrstuvwxyI,%\QRcode \par \QRversion\box\QRcodebox\par\vfil\eject
\QRdataadd,abcdefghijklmnopqrstuvwxyzaJ,%\QRcode \par \QRversion\box\QRcodebox\par\vfil\eject
\QRdataadd,abcdefghijklmnopqrstuvwxyzabcdK,%\QRcode \par \QRversion\box\QRcodebox\par\vfil\eject
\QRdataadd,abcdefghijklmnopqrstuvwxyzabcdefgijklmnL,%\QRcode \par \QRversion\box\QRcodebox\par\vfil\eject
\QRdataadd,abcdefghijklmnopqrstuvwxyzabcdefghijklmnM,%\QRcode \par \QRversion\box\QRcodebox\par\vfil\eject
\QRdataadd,abcdefghijklmnopqrstuvwxyzabcdefghiN,%\QRcode \par \QRversion\box\QRcodebox\par\vfil\eject
\QRdataadd,abcdefghijklmnopqrstuvwxyzabcdefghijklmnopqrO,%\QRcode \par \QRversion\box\QRcodebox\par\vfil\eject
\QRdataadd,abcdefghijklmnopqrstuvwxyzabcP,%\QRcode% \par \QRversion\box\QRcodebox\par\vfil\eject
\QRdataadd,abcdefghijklmnopqrstuvwxyzabcdefghijklmnopqrQ,%\QRcode% \par \QRversion\box\QRcodebox\par\vfil\eject
\QRdataadd,abcdefghijklmnopqrstuvwxyzabcdefghijklmnopqrR,%\QRcode% \par \QRversion\box\QRcodebox\par\vfil\eject
\QRdataadd,abcdefghijklmnopqrstuvwxyzabcdS,%\QRcode% \par \QRversion\box\QRcodebox\par\vfil\eject
\QRdataadd,abcdefghijklmnopqrstuvwxyzabcdefghijT,%\QRcode \par \QRversion\box\QRcodebox\par\vfil\eject
\QRdataadd,abcdefghijklmnopqrstuvwxyzabcdefghijklmnopqrU,%\QRcode% \par \QRversion\box\QRcodebox\par\vfil\eject
\QRdataadd,abcdefghijklmnopqrstuvwxyzabcdefghijklmnopqrV,%\QRcode \par \QRversion\box\QRcodebox\par\vfil\eject
\QRdataadd,abcdefghijklmnopqrstuvwxyzabcdefghijklmnopqrW,%\QRcode \par \QRversion\box\QRcodebox\par\vfil\eject
\QRdataadd,abcdefghijklmnopqrstuvwxyzabcdefghijklmnopqrX,%\QRcode \par \QRversion\box\QRcodebox\par\vfil\eject
\QRdataadd,abcdefghijklmnopqrstuvwxyzabcdefghijklmnopqrY,%\QRcode \par \QRversion\box\QRcodebox\par\vfil\eject
\QRdataadd,abcdefghijklmnopqrstuvwxyzabcdefghijklmnopqrZ,%\QRcode \par \QRversion\box\QRcodebox\par\vfil\eject
\QRdataadd,abcdefghijklmnopqrstuvwxyzabcdefghijklmnopqr36,%\QRcode \par \QRversion\box\QRcodebox\par\vfil\eject
\QRdataadd,abcdefghijklmnopqrstuvwxyzabcdefghijklmnopqr37,%\QRcode \par \QRversion\box\QRcodebox\par\vfil\eject
\QRdataadd,abcdefghijklmnopqrstuvwxyzabcdefghijklmnopqr38,%\QRcode \par \QRversion\box\QRcodebox\par\vfil\eject
\QRdataadd,abcdefghijklmnopqrstuvwxyzabcdefghijklmnopqr39,%\QRcode \par \QRversion\box\QRcodebox\par\vfil\eject
\QRdataadd,abcdefghijklmnopqrstuvwxyzabcdefghijklmnopqrstuvwxyzabcdefghijklmnopqrst40,\QRcode \par\vfil\eject \QRversion\box\QRcodebox\par\vfil\eject
\box\QRcodebox\par

\def\QRLogMask{\QRnumA=5 \QRLogMaskLoop
\!\show\csname QRecMask\QRMask\EClevel E\endcsname
\!\show\csname QRecMask\QRMask\EClevel D\endcsname
\!\show\csname QRecMask\QRMask\EClevel C\endcsname
\!\show\csname QRecMask\QRMask\EClevel B\endcsname
\!\show\csname QRecMask\QRMask\EClevel A\endcsname
\!\show\csname QRecMask\QRMask\EClevel 9\endcsname
\!\show\csname QRecMask\QRMask\EClevel 8\endcsname
\!\show\csname QRecMask\QRMask\EClevel 7\endcsname
\!\show\csname QRecMask\QRMask\EClevel 6\endcsname
\!\show\csname QRecMask\QRMask\EClevel 5\endcsname
\!\show\csname QRecMask\QRMask\EClevel 4\endcsname
\!\show\csname QRecMask\QRMask\EClevel 3\endcsname
\!\show\csname QRecMask\QRMask\EClevel 2\endcsname
\!\show\csname QRecMask\QRMask\EClevel 1\endcsname
\!\show\csname QRecMask\QRMask\EClevel 0\endcsname
}%
\def\QRLogMaskLoop{\QRnumB=5 \QRLogMaskLoopLoop\advance\QRnumA by -1 \ifnum\QRnumA<0 \else\!\QRLogMaskLoop\fi}%
\def\QRLogMaskLoopLoop{\QRLogMaskLoopLoopDo\advance\QRnumB by -1 \ifnum\QRnumB<0 \else\!\QRLogMaskLoopLoop\fi}%
\def\QRLogMaskLoopLoopDo{\!\show\csname QRmasked\the\QRnumA,\the\QRnumB\endcsname
\!\show\csname QRmasked\QRMask\the\QRnumA,\the\QRnumB\endcsname}%
\bye
