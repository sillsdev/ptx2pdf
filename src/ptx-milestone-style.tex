%:strip
% Part of the ptx2pdf macro package for formatting USFM text
% copyright (c) 2007-2020 by SIL International
% written by David Gardner and pior editors of ptx-char-style
%
% Permission is hereby granted, free of charge, to any person obtaining  
% a copy of this software and associated documentation files (the  
% "Software"), to deal in the Software without restriction, including  
% without limitation the rights to use, copy, modify, merge, publish,  
% distribute, sublicense, and/or sell copies of the Software, and to  
% permit persons to whom the Software is furnished to do so, subject to  
% the following conditions:
%
% The above copyright notice and this permission notice shall be  
% included in all copies or substantial portions of the Software.
%
% THE SOFTWARE IS PROVIDED "AS IS", WITHOUT WARRANTY OF ANY KIND,  
% EXPRESS OR IMPLIED, INCLUDING BUT NOT LIMITED TO THE WARRANTIES OF  
% MERCHANTABILITY, FITNESS FOR A PARTICULAR PURPOSE AND  
% NONINFRINGEMENT. IN NO EVENT SHALL SIL INTERNATIONAL BE LIABLE FOR  
% ANY CLAIM, DAMAGES OR OTHER LIABILITY, WHETHER IN AN ACTION OF  
% CONTRACT, TORT OR OTHERWISE, ARISING FROM, OUT OF OR IN CONNECTION  
% WITH THE SOFTWARE OR THE USE OR OTHER DEALINGS IN THE SOFTWARE.
%
% Except as contained in this notice, the name of SIL International  
% shall not be used in advertising or otherwise to promote the sale,  
% use or other dealings in this Software without prior written  
% authorization from SIL International.
%%%%%%%%%%%%%%%%%%%%%%%%%%%%%%%%%%%%%%%%%%%%%%%%%%%%%%%%%%%%%%%%%%%%%%%

% Milestone macros
\def\mst@nestyle#1{\trace{s}{mst@nestyle:#1}%
 \gdef\thismil@stone{\detokenize{#1}}% record the name of the style
 \catcode32=12 % make <space> an "other" character, so it won't be skipped by \futurelet
 \catcode13=12 % ditto for <return>
 \futurelet\n@xt\domst@nestyle % look at following character and call \domst@nestyle
}

\catcode`\~=12 \lccode`\~=32 % we'll use \lowercase{~} when we need a category-12 space
\catcode`\_=12 \lccode`\_=13 % and \lowercase{_} for category-12 <return>
\lowercase{
 \def\domst@nestyle{% here, \n@xt has been \let to the next character after the marker
  \catcode32=10 % reset <space> to act like a space again
  \catcode13=10 % and <return> is also a space (we don't want blank line -> \par)
  \if\n@xt-\let\n@xt@\startmst@nestyle@minus\else
    \if\n@xt~\let\n@xt@\startmst@nestyle@spc\else
      \if\n@xt_\let\n@xt@\startmst@nestyle@nl\else
	\let\n@xt@\startmst@nestyle\fi\fi\fi
  \n@xt@
 }
 \def\startmst@nestyle@spc~{\startmst@nestyle}%                                             
 \def\startmst@nestyle@nl_{\startmst@nestyle}%
 \def\startmst@nestyle@minus-#1{\mst@netrue\edef\thismil@stoneVal{}\xdef\milestoneOp{#1}\startmst@nestyle}%
}
\def\*{\ifmst@nefalse\endch@rstyle*\fi\proc@ttribs\mil@stone}% USFM3 'self closing marker' milestone
\newif\ifmst@ne
\def\mil@stone{\trace{s}{Milestone \newch@rstyle (\milestoneOp)}%
  \if\milestoneOp s\relax\st@rtmilestone\else
   \if\milestoneOp e\relax\@ndmilestone\else
    \st@ndalonemilestone\fi\fi
 \let\milestoneOp\empty
 \s@tfont{\thisch@rstyle}% set up font attributes
}

\def\startmst@nestyle{%What actually changes between the beginning of a milestone and the end-marker? 
   \init@ttribs
}

\newif\ifin@ttrib
\def\init@ttribs{
 % Attribute code goes here.
  \def\attrst@ck{,}%
  \in@ttribfalse
  \catcode`\|=\active
}
\def\proc@ttribs{
  \catcode`\|=12
  \ifin@ttrib\egroup
    \trace{A}{Attributes specified:\attrib@rgs}%
  \fi
}
{
  \catcode`\|=\active
  \def|{\in@ttribtrue\def\attrib@rgs\bgroup}
}

\def\dr@pmilest@ne#1+#2\E{% This kills a milestone that might be quite burried in the stack
  \edef\MSt@mp{#2}
  \if#1m\relax
    \ifx\MSt@mp\mcch@cking\relax
      %\csname endit@#1\endcsname
      \trace{s}{Found #1+#2}%
    \else
      \ifx\@ut\empty\xdef\@ut{#1+#2}\else\xdef\@ut{\@ut,#1+#2}\fi 
    \fi
  \else
    \ifx\@ut\empty\xdef\@ut{#1+#2}\else\xdef\@ut{\@ut,#1+#2}\fi 
  \fi
  \trace{s}{dr@pmilest@ne: \@ut}%
}
\def\dr@pmilestone#1{
  \trace{s}{Dropping milestone #1 from \MScstack}%
  \edef\mcch@cking{#1}%
  \let\@ut=\empty
  \let\d@=\dr@pmilest@ne
  \MScdown
  \let\MScstack=\@ut
  \trace{s}{Stack now: \MScstack}%
}
